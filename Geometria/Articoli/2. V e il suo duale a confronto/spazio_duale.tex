\documentclass[a4paper]{article}
\usepackage[utf8]{inputenc}
\usepackage[italian]{babel}
\usepackage{amsfonts}
\usepackage{amsthm}
\usepackage{amssymb}
\usepackage{amsopn}
\usepackage{mathtools}
\usepackage{marvosym}

\newcommand{\dual}[1]{#1^{*}}

\title{$V$ e $\dual V$ a confronto}
\author{Gabriel Antonio Videtta}
\date{\today}

\DeclareMathOperator{\tr}{tr}
\DeclareMathOperator{\Ker}{Ker}
\DeclareMathOperator{\Imm}{Im}

\setlength\parindent{0pt}

\begin{document}

\maketitle

\newcommand{\BB}{\mathcal{B}}
\newcommand{\FF}{\mathbb{F}_2}
\newcommand{\NN}{\mathbb{N}}
\newcommand{\ZZ}{\mathbb{Z}}
\newcommand{\KK}{\mathbb{K}}
\newcommand{\LL}[2]{\mathcal{L} \left(#1, \, #2\right)}

\newcommand{\MM}[2]{\mathcal{M}_{#1 \times #2}\left(\KK\right)}
\newcommand{\M}[1]{\mathcal{M}_{#1}\left(\KK\right)}
\newcommand{\Mbb}[3]{\mathcal{M}^{#1}_{#2} \left( #3 \right)}
\newcommand{\Mb}[2]{\mathcal{M}^{#1}_{#2}}

\theoremstyle{definition}
\newtheorem{definition}{Definizione}[section]

\renewcommand{\vec}[1]{\underline{#1}}

\newtheorem{example}{Esempio}[section]
\newtheorem{exercise}{Esercizio}[section]
\newtheorem{theorem}{Teorema}[section]
\newtheorem{proposition}{Proposizione}[section]
\newtheorem{corollary}{Corollario}[section]

\tableofcontents

\section{Premessa e motivazione}

Lo studio delle applicazioni lineari è riconosciuto come uno
degli aspetti fondamentali della geometria contemporanea. Non è
infatti una mera coincidenza che nella maggior parte delle
applicazioni impiegate nello studio dei sistemi lineari si
riconoscano proprietà che sono proprie delle applicazioni lineari. \\

Uno dei primi esempi importanti di applicazione lineare è
quello della funzione traccia $\tr : \M{n} \to \KK$, che
associa a una matrice quadrata la somma degli elementi della
diagonale principale. Un altro esempio è quello del determinante
$\det : \left(\KK^{n}\right)^n \to \KK$, un'applicazione che generalizza
il concetto di linearità a più argomenti. Si parla infatti in
questo caso di un'applicazione multilineare. \\

In ogni caso, queste due importanti applicazioni sono
accomunate dallo spazio di arrivo, il campo $\KK$, sul
quale si fonda lo spazio di partenza. Per approfondire lo
studio di questo tipo di applicazioni, si introduce
pertanto il concetto di \textbf{spazio duale}.

\section{Lo spazio duale e le sue proprietà}

\begin{definition}
Si dice \textbf{spazio duale} di uno spazio vettoriale $V$
lo spazio delle applicazioni lineari $\LL{V}{\KK}$, indicato con $\dual V$.
\end{definition}

\subsection{Il caso finito}

Prima di dedurre la dimensione e una base ``naturale'' di
$\dual V$, introduciamo il seguente teorema, che mette
in correlazione due spazi apparentemente scollegati.

\vskip 10pt

\begin{theorem} Siano $V$ e $W$ due spazi vettoriali di dimensione finita su $\KK$, e siano $\dim V = n \in \NN$, $\dim W = m \in \NN$. Allora $\LL{V}{W} \cong \MM{m}{n}$.
\label{isom}
\end{theorem}

\begin{proof}
Siano $\BB = \left(\vec v_1, \, \dots, \, \vec v_n\right)$ e
$\BB' = \left(\vec w_1, \, \dots, \, \vec w_m\right)$ 
basi ordinate rispettivamente di $V$ e di $W$. \\

Si considera l'applicazione lineare
$\Mb{\BB}{\BB'} : \LL{V}{W} \to \MM{m}{n}$, che
associa ad ogni applicazione lineare la sua matrice di
cambiamento di base. \\

Tale applicazione è iniettiva, dal momento che l'unica
applicazione a cui è associata la matrice nulla è l'applicazione
che associa ad ogni vettore lo zero di $W$. \\

Inoltre, $\Mb{\BB}{\BB'}$ è surgettiva, poiché data una
matrice $\mathbf{m} \in \MM{m}{n}$ si può costruire l'applicazione
$\phi : V \to W$ t.c. $\left[ \phi \left( v_i \right) \right]_{\BB'} = \mathbf{m}^i \; \forall \, i \in \NN \mid 1 \leq i \leq n$. \\

Dal momento che $\Mb{\BB}{\BB'}$ è sia iniettiva che
surgettiva, tale applicazione è bigettiva, e quindi
un isomorfismo.
\end{proof}

\vskip 10pt

\begin{corollary}
Sia $V$ uno spazio vettoriale di dimensione finita su $\KK$.
Allora $\dim V = \dim \dual V$.
\label{isom2}
\end{corollary}

\begin{proof}
Dal \textit{Teorema \ref{isom}} si deduce che
$\dim \dual V = \dim \LL{V}{\KK} = \dim \KK \cdot \dim V =
\dim V$.
\end{proof}

\vskip 10pt

\begin{corollary}
Sia $V$ uno spazio vettoriale di dimensione finita su $\KK$.
Allora $V \cong \dual V$.
\label{isom3}
\end{corollary}

\begin{proof}
Poiché $V$ è di dimensione finita, la dimostrazione segue dal
\textit{Corollario \ref{isom2}}, dal momento che
$\dim V = \dim \dual V \iff V \cong \dual V$.
\end{proof}

\begin{proof}[Dimostrazione alternativa.]
Sia $\dim V = n \in N$ e sia $\BB = \left(\vec v_1, \, \dots, \, \vec v_n\right)$ una base ordinata di $V$. \\

Si costruisce un'applicazione $\phi : V \to \dual V$ che,
detto $\vec v = \sum_{i=0}^{n} \alpha_i \, \vec v_i$ con
$\alpha_i \in \KK \; \forall \, i \in \NN \mid 1 \leq i \leq n$, sia tale che:

\[\phi(\vec v) = \sum_{i=0}^{n} \alpha_i \, \vec v_i^{*}\]

con $\vec v_i^{*}$ costruito nel seguente modo\footnote{Si sarebbe potuto
semplificare la grafia introducendo la notazione del \textit{delta di Dirac}, ossia $\delta_{ij}$. Si è
tuttavia preferito esplicitare la definizione del funzionale.}:

\[\vec v_{i}^*\left(\vec v_j\right) = \begin{cases}1 & \text{se } i = j \\ 0 & \text{altrimenti} \end{cases}\]

L'applicazione $\phi$ è chiaramente lineare. Poiché i vari
$\vec v_i^{*}$ sono linearmente indipendenti, segue che
$\Ker \phi = \{\vec 0\}$, e quindi che $\phi$ è iniettiva. \\

Sia $\xi \in \dual V$. Allora $\xi \left( \vec v \right) =
\sum_{i=1}^{n} \alpha_i \, \xi(\vec v_i) =
\sum_{i=1}^{n} \vec v_i^{*} \left( \vec v \right) \xi(\vec v_i) \, $. Quindi $\xi = \sum_{i=1}^n \xi(\vec v_i) \, \vec v_i^{*}$.
Detto $\vec u = \sum_{i=1}^n \xi \left( \vec v_i \right) \vec v_i$, si verifica che $\phi \left( \vec u \right) = \xi$.
Pertanto $\phi$ è surgettiva\footnote{Alternativamente, per il teorema del rango, $\dim V$ = $\dim \Imm \phi + \underbrace{\dim \Ker \phi}_{=\,0} = \dim \Imm \phi \implies \dim \Imm \phi = \dim V = \dim \dual V \implies \Imm \phi = \dual V$, ossia che $\phi$ è surgettiva.}. \\

Poiché iniettiva e surgettiva, $\phi$ è bigettiva, e pertanto
un isomorfismo.

\end{proof}

\vskip 10pt

\begin{corollary} Sia $V$ uno spazio vettoriale di dimensione
finita su $\KK$, con $\dim V = n \in \NN$.
L'insieme $\dual \BB = \left( \vec v_i^{*} \right)_{i=1\to n}$ è una base di $\dual V$.
\end{corollary}

\begin{proof}
Dal \textit{Corollario \ref{isom3}} si desume che la dimensione
di $\dual V$ è esattamente $n$. Poiché $\dual \BB$ è un insieme
linearmente indipendente di $n$ elementi, si conclude
che è una base di $\dual V$.
\end{proof}

\subsection{Il caso infinito}

Le dimostrazioni presentate precdentemente non prendono in considerazione
il caso degli spazi vettoriali di dimensione infinita;
ciononostante vale in particolare un risultato correlato:

\vskip 10pt

\begin{theorem} Sia $V$ uno spazio vettoriale su $\KK$.
$\dim V = \infty \iff \dim \dual V = \infty$\footnote{Ciò
tuttavia non implica che $V$ e $\dual V$ siano equipotenti se di dimensione infinita; al contrario, $| \dual V | > |V|$.}.
\end{theorem}

\begin{proof}
Se $\dual V$ è di dimensione infinita, anche $V$ deve esserlo
necessariamente, altrimenti, per il \textit{Teorema \ref{isom}}
dovrebbe esserlo anche $\dual V$. \\

Sia allora $V$ di dimensione infinita e sia $A_i$ una famiglia
di indici che enumeri $i$ elementi della base di $V$.

Si consideri l'insieme linearmente indipendente $I_n = \{\vec v_{\alpha}^* \}_{\alpha \in A_n}$ con\footnote{Ancora una volta
questa definizione ricalca il delta di Dirac.}:

\[\vec v_{\alpha}^*\left(\vec v_{\beta}\right) = \begin{cases}1 & \text{se } \alpha = \beta \\ 0 & \text{altrimenti} \end{cases}\]

Si assuma l'esistenza di una base $\BB$ di $\dual V$ di
cardinalità finita, e sia $| \BB | = n \in \NN$. Ogni insieme $P \subset V$ linearmente indipendente è t.c. $|P| \leq n$.
Tuttavia $|I_{n+1}|=n+1>n$, \Lightning.

\end{proof}

\section{Esercizi}

\begin{exercise}
Si dimostri che l'insieme $I_n$ è linearmente indipendente in $\dual V$, dato $V$ spazio vettoriale di dimensione infinita.
\end{exercise}

\begin{exercise}
Dato $V$ uno spazio vettoriale di dimensione finita, si esibisca
una base per $\dual {\left( \dual V \right)} = \LL{\LL{V}{ \KK}}{\KK}$, il cosiddetto \textbf{spazio biduale}.
\end{exercise}

\end{document}
