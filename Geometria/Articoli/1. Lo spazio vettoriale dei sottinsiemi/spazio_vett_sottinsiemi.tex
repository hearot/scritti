\documentclass[a4paper]{article}
\usepackage[utf8]{inputenc}
\usepackage[italian]{babel}
\usepackage{amsfonts}
\usepackage{amsthm}
\usepackage{amssymb}
\usepackage{amsopn}
\usepackage{mathtools}

\title{Spazio vettoriale dei sottoinsiemi}
\author{Gabriel Antonio Videtta}
\date{\today}

\begin{document}

\maketitle

\newcommand{\FF}{\mathbb{F}_2}
\newcommand{\ZZ}{\mathbb{Z}}

\newtheorem{example}{Esempio}[section]
\newtheorem{exercise}{Esercizio}[section]
\newtheorem{theorem}{Teorema}[section]

\setlength\parindent{0pt}

\tableofcontents

\section{Il campo $\FF$ e verifica degli assiomi}

Dato un qualsiasi insieme X\footnote{Non ci soffermiamo sulla definizione di insieme,
sebbene da tale scelta possano scaturire vari paradossi. Rimandiamo per la risoluzione
di tali problemi a varie teorie assiomatiche, come quella di Zermelo–Fraenkel.} è possibile
estrarne uno spazio vettoriale. \\ \\
Per costruire l'insieme di vettori considereremo gli elementi di X, mentre il campo su cui verrà
costruito lo spazio sarà $\FF = \{0,1\} \, \cong \, \ZZ/2\ZZ$. Prima di costruire lo spazio,
assicuriamoci che $\FF$ sia effettivamente un campo\footnote{Non solo è un campo, ma è il più
piccolo campo non banale, ossia con più di un elemento.} e definiamolo. \\ \\
Le operazioni $+$ e $\cdot$ di questo campo sono esattamente le stesse impiegate in
$\ZZ/2\ZZ$ (i.e. in modulo 2, dove $2 \equiv 0$), o equivalentemente vengono definite in questo modo:
\begin{itemize}
    \item $+ : \FF \to \FF$ t.c. $0+0=0$, \, $1+0=1$, \, $0+1=1$, \, $1+1=0$ (addizione)
    \item $\cdot : \FF \to \FF$ t.c. $0\cdot0=0$, \, $1\cdot0=0$, \, $0\cdot1=0$, \, $1\cdot1=1$ (moltiplicazione)
\end{itemize}
Gli assiomi di campo sono effettivamente soddisfatti: gli inversi additivi di $0$ e $1$ sono $0$ e $1$ stessi, e $1$ è inverso moltiplicativo di sé stesso. Valgono chiaramente le
proprietà associative e distributive, mentre gli elementi neutri sono $0$ per l'addizione e
$1$ per la moltiplicazione.
Adesso è possibile costruirci sopra uno spazio vettoriale, che d'ora in poi chiameremo
$\Delta (X)$.

\section{Costruzione dello spazio vettoriale}

Ricordiamo una delle operazioni elementari degli insiemi, la cosiddetta \textit{differenza
simmetrica} $A \Delta B$. Essa altro non è che l'unione dei due insiemi tolta la loro
intersezione: \[A \Delta B = \left( A \cup B \right) \setminus \left( A \cap B \right).\]
Adesso definiamo $\Delta (X) = \{ \alpha_1 x_1 + \alpha_2 x_2 + \ldots + \alpha_n x_n \mid n \in \mathbb{N} \land x_i \in X, \alpha_i \in \FF \; \forall \, i \in \mathbb{N} \mid  1 \leq i \leq n \}$.

\begin{example}
Se $X = \{a, b\}$, $\Delta (X) = \{0, \, a, \, b, \, a+b\}$.
\end{example}
\begin{example}
Se $X = \{a, b, c\}$, $\Delta (X) = \{0, \, a, \, b, \, c, \, a+b, \, a+c, \, b+c, \, a+b+c\}$.
\end{example}

Dotiamo lo spazio di due operazioni, dette somma ($+$) e prodotto esterno ($\cdot$):

\begin{itemize}
    \item $+ : \Delta (X) \to \Delta (X)$ t.c. $\forall \, a, b \in \Delta (X), \, a+b$ sia il risultato della
    somma coefficiente a coefficiente\footnote{Esattamente come accade nei polinomi, dove
    la somma di due polinomi è effettuata sommando i coefficienti dei monomi dello stesso
    grado. L'unica differenza risiede nel ricordarsi che la somma dei coefficienti in $\Delta (X)$ è quella di $\FF$, dove il caso $1+1=0$ ha particolare rilevanza.}.
    \item $\cdot : \Delta (X) \to \Delta (X)$ t.c. $\forall \, a \in \Delta (X), \, \delta \in \FF, \, \delta a$ sia il risultato del prodotto di $\delta$ con ogni coefficiente di $a$\footnote{Sussiste ancora l'analogia con i polinomi.}.
\end{itemize}

\begin{example}
Se $X = \{a, \, b\}$, $a+a=(1+1)a=0$ in $\Delta (X)$, mentre $a+b$ ''rimane'' $a+b$.
\end{example}

\begin{example}
Se $X = \{a, \, b\}$, $1\cdot a=a$ in $\Delta (X)$, mentre $0 \cdot a = 0$.
\end{example}

Queste operazioni verificano facilmente gli assiomi dello spazio vettoriale, pertanto $\Delta (X)$ è uno spazio vettoriale, la cui base è $X$ stesso\footnote{Ogni elemento di $\Delta (X)$ è infatti combinazione lineare univoca degli elementi di $X$ -- ancora una volta, come accade nei polinomi.}. \\

Pertanto $\dim \Delta (X) = |X|$, se $|X| < \infty$, altrimenti $\dim \Delta (X) = \infty$. \\

L'interpretazione (e l'utilità) di questo spazio è facilmente spiegata: ogni elemento di
$\Delta (X)$ definisce in modo univoco un sottoinsieme di $X$ e l'operazione definita
altro non è che la differenza simmetrica $A \Delta B$ ricordata all'inizio della sezione. \\

In altri termini, dotando dell'insieme delle parti (i.e. dei sottoinsiemi) di $X$, detto
$\wp(X)$, dell'operazione differenza simmetrica per l'addizione e dell'operazione esistenza\footnote{$1 \cdot X = X$, $0 \cdot X = \varnothing$.} per il prodotto esterno, si può verificare che questo costituisce
uno spazio vettoriale su $\FF$ isomorfo a $\Delta (X)$ nel caso in cui $X$ sia un insieme finito\footnote{L'isomorfismo impiegato nella dimostrazione difatti non è definito per i sottoinsiemi infiniti -- dopotutto, se $|X| = \infty$, $\Delta (X)$ è un insieme numerabile, mentre $\wp(X)$ non può esserlo.}.

\begin{theorem}
$|X| < \infty \Rightarrow \wp(X) \cong \Delta (X)$
\end{theorem}

\begin{proof}
Per dimostrare che i due spazi sono isomorfi si costruisce un'applicazione lineare
bigettiva. Definiamo pertanto $\phi : \wp(X) \to \Delta (X)$ in modo tale che:

\[\phi(\{a_1, \, \ldots, \, a_n\}) = a_1 + \ldots + a_n\]

Dimostriamo che $\phi$ è un'applicazione lineare, dimostrandone prima la linearità e poi
l'omogeneità. \\

Verifichiamo la linearità:
\[\begin{split}
&\phi (\{a_1, \, \ldots, \, a_n, \, b_{n+1}, \, \ldots, \, b_m \} \Delta \{a_1, \, \ldots, \, a_n, \, c_{n+1}, \, \ldots, \, c_k \}) = \\
&\;\;\;= \phi (\{ b_{n+1}, \, \ldots, \, b_m, \, c_{n+1}, \, \ldots, \, c_k \}) = \\
&\;\;\;= b_{n+1} + \ldots + b_m + \ldots + c_{n+1} + \ldots + c_k = \\
&\;\;\;= (a_1 + \ldots + b_{n+1} + \ldots + b_m) + (a_1 + \ldots + c_{n+1} + \ldots + c_k) = \\
&\;\;\;= \phi (\{a_1, \, \ldots, \, a_n, \, b_{n+1}, \ldots, \, b_m \}) + \phi (\{a_1, \, \ldots, \, a_n, \, c_{n+1}, \ldots, \, c_k \}).
\end{split}\]

E l'omogeneità con $1$:
\[\phi (1 \cdot \{a_1, \, \ldots, \, a_n \}) = \phi (\{a_1, \, \ldots, \, a_n \}) =  1 \cdot \phi (\{a_1, \, \ldots, \, a_n \}).\]

Ed infine con $0$:

\[\phi (0 \cdot \{a_1, \, \ldots, \, a_n \}) = \phi (\varnothing) = 0 = 0 \cdot \phi (\{a_1, \, \ldots, \, a_n \}).\]

Questa applicazione è iniettiva, dal momento che $\operatorname{Ker}\phi = \{\varnothing\}$.
Inoltre $\phi$ è surgettiva, dal momento che una controimmagine di un elemento $d$ di $\Delta (X)$ è l'insieme delle parti letterali di $d$. \\

Poiché bigettiva, tale applicazione è un isomorfismo.

\end{proof}

\section{Note ed esercizi}

In realtà, è possibile costruire un'infinità di spazi su $X$ mantenendo le stesse operazioni, ma variando il campo su cui esso
è costruito. Un caso speciale, che merita una
menzione onorevole, è proprio $X = \{1, \, x, \, x^2, \, \ldots\}$ costruito su $\mathbb{R}$ (o un qualsiasi
$\mathbb{K}$ campo), che dà vita allo spazio dei polinomi, detto $\mathbb{R}[x]$ (o
$\mathbb{K}[x]$).

\begin{exercise}
Si esibisca un controesempio per la dimostrazione dell'isomorfismo nel caso infinito.
\end{exercise}

\begin{exercise}
Si dimostri che, se $X$ è finito, anche $\{ x_1, \, x_1+x_2, \, \ldots, \, \sum_{i=1}^{|X|} x_i \}$ con $x_i$ elementi distinti
di $X$ è una base di $\Delta (X)$.
\end{exercise}

\begin{exercise}
Dopo aver mostrato che $\{ 1, \, x, \, x^2 \, , \, \ldots\}$ è una base di $\mathbb{R}[x]$\footnote{Questa particolare base è detta \textit{base standard} di $\mathbb{R}[x]$.},
si dimostri che anche $\{ \sum_{i=0}^{j} x^i \mid j \in \mathbb{N}, \, j \geq 0 \}$, con $x^0 = 1$, lo è.
\end{exercise}

\end{document}
