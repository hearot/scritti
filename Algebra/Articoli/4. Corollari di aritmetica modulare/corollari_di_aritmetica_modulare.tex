\documentclass[12pt]{scrartcl}
\usepackage{notes_2023}

\begin{document}
	\title{Corollari di aritmetica modulare}
	\date{\today}
	\maketitle
	
	In questo breve documento dimostro ognuno
	dei seguenti teoremi di teoria dei numeri:
	
	\begin{enumerate}[(i)]
		\item il teorema di Wilson,
		\item il teorema di Wolstenholme,
		\item il teorema di Lagrange (per i polinomi).
	\end{enumerate}
	
	Questi teoremi si dimostrano facilmente anche senza l'uso dei risultati
	principali della teoria dei gruppi e degli anelli. Tuttavia, la
	loro vera natura è prettamente dovuta allo studio di queste due
	teorie -- come si evince dalla brevità e dall'immediatezza delle
	dimostrazioni. \bigskip
	
	
	
	Il prerequisito fondamentale per approcciare questi tre teoremi
	è l'aver studiato il teorema di Lagrange di teoria dei gruppi
	(quantomeno per dimostrare i primi due teoremi)\footnote{
		Oppure il suo più semplice corollario, il piccolo teorema di Fermat.
	} e avere familiarità con gli anelli euclidei (per dimostrare
	il teorema di Lagrange). \bigskip
	
	
	Si presenta innanzitutto il seguente lemma:

	\begin{lemma}
		Sia $p$ un numero primo.
		Sia $q \in \ZZ[x]$ il polinomio tale per cui:
		\[
			q(x) = (x-1)(x-2)\cdots(x-(p-1)).
		\]
		Allora, se
		\[
			q(x) = x^{p-1} + a_{p-2} x^{p-2} + \ldots + a_1 x + a_0, \qquad a_i \in \ZZ,\;0 \leq i \leq p-1,
		\]
		vale che $\hat q(x) = x^{p-1} -1$, dove $\hat q$ è la proiezione in $\ZZ \quot p\ZZ$ di $q$.
	\end{lemma}
	
	\begin{proof}
		Per il teorema di Lagrange, vale che
		$x^{p-1} - 1 \equiv 0 \pod p$ per ogni $x \in \ZZ \quot p\ZZ^*$, dal momento che
		$\ZZ \quot p\ZZ^*$ è un gruppo moltiplicativo di ordine $p-1$.
		Pertanto $\hat q$ e $x^{p-1} - 1$ hanno le stesse radici e
		lo stesso grado, e sono dunque lo stesso polinomio, da cui la tesi.
	\end{proof}
	
	\begin{theorem}[di Wilson]
		Sia $p \in \NN^+$. Allora $(p-1)! \equiv -1 \pod p$ se e solo se $p$ è primo.
	\end{theorem}
	
	\begin{proof}
		Se $(p-1)! \equiv -1 \pod p$, allora ogni elemento di $\ZZ \quot p \ZZ$ è
		invertibile, e quindi $\ZZ \quot p \ZZ$ sarebbe un campo; ciò è possibile
		se e solo se $p$ è primo\footnote{
			Infatti un campo è prima di tutto un dominio. Se $p$ non fosse
			primo, $\ZZ \quot p \ZZ$ ammetterebbe divisori di zero.
		}. Se $p$ è primo, per il \textit{Lemma 1}, $\hat q(x) = x^{p-1} - 1$, e quindi
		$p$ divide ogni $a_i$. Si
		osserva allora che $a_0 = (-1)^{p-1} (p-1)!$ e che deve dunque valere:
		\[ (-1)^{p-1} (p-1)! \equiv -1 \pod p. \]
		Sia che $p$ sia uguale a $2$, sia che $p$ sia dispari, l'ultima equazione
		implica che:
		\[ (p-1)! \equiv -1 \pod p, \]
		da cui la tesi.
 	\end{proof}
 	
 	\begin{theorem}[di Wolstenholme]
 		Sia $p \geq 5$ un numero primo. Allora il numeratore di:
 		\[
 			S = 1 + \frac{1}{2} + \ldots + \frac{1}{p-1}
 		\]
 		è divisibile per $p^2$.
 	\end{theorem}
 	
 	\begin{proof}
 		Per il \textit{Lemma 1}, $p$ divide ogni $a_i$ di $q$. Si osserva
 		inoltre che $a_1$ è esattamente il numeratore di $S$. \medskip
 

 		Si computa $q$ in $p$:
 		\[ 
 			q(p) = p^{p-1} + a_{p-2} p^{p-2} + \ldots + a_1 p + a_0.
 		\]
 		Analogamente:
 		\[
 			q(p) = (p-1)(p-2) \cdots (p-(p-1)) = (p-1)! = (-1)^{p-1} a_0 = a_0,
 		\]
 		dove si è usato che $p \geq 5$ è dispari. Quindi
 		vale che:
 		\[
 			p^{p-1} + a_{p-2} p^{p-2} + \ldots + a_1 p = 0,
 		\]
 		da cui:
 		\[
 			a_1 p = -(p^{p-1} + \ldots + a_2 p^2).
 		\]
 		Poiché $p > 3$, $p^3$ divide il secondo membro dell'equazione, e quindi
 		$p^2$ divide $a_1$, da cui la tesi.
 	\end{proof}
 	
 	\begin{theorem}[di Lagrange, per i polinomi]
 		Sia $q$ un polinomio in $\ZZ[x]$ e sia $p$ un numero primo.
 		Allora vale una delle seguenti due affermazioni:
 
 		\begin{itemize}
 			\item $p$ divide ogni coefficiente di $q$,
 			\item esistono al più $\deg q$ soluzioni incongruenti\footnote{
 				Due soluzioni $x$, $y \in \ZZ$ si dicono incongruenti se
 				$x \not\equiv y \pod p$.
 			} di $q$ in
 				$\ZZ \quot p \ZZ$.
 		\end{itemize}
 	\end{theorem}
 	
 	\begin{proof}
 		Si consideri la proiezione in $\ZZ \quot p \ZZ$ di $q$, indicata
 		con $\hat q$. Poiché $p$ è primo, $\ZZ \quot p \ZZ$ è un campo,
 		e quindi $\ZZ \quot p \ZZ[x]$ è un anello euclideo. Pertanto,
 		se $\hat q$ è diverso da $0$, $\hat q$ ammette al più
 		$\deg \hat q$ soluzioni in $\ZZ \quot p \ZZ$. In particolare
 		vale che $\deg \hat q \leq \deg q$, e quindi $\hat q$ ammette al
 		più $\deg q$ soluzioni in $\ZZ \quot p \ZZ$ (e quindi esistono al più
 		$\deg q$ classi di resto che sono soluzione in $\ZZ \quot p \ZZ$). Se
 		invece $\hat q = 0$, $p$ deve dividere obbligatoriamente
 		ogni coefficiente di $q$, da cui la tesi.
 	\end{proof}
\end{document}