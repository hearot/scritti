\documentclass[a4paper]{article}
\usepackage[utf8]{inputenc}
\usepackage[italian]{babel}
\usepackage{amsfonts}
\usepackage{amsthm}
\usepackage{amssymb}
\usepackage{amsopn}
\usepackage{bm}
\usepackage{mathtools}
\usepackage{marvosym}
\usepackage{tikz}
\usepackage{wrapfig}
\usepackage{xpatch}

\usepackage{algorithm2e}

\usepackage{csquotes}
\usepackage{biblatex}

\addbibresource{bibliography.bib}

\usepackage[colorlinks=false,bookmarksopen=true]{hyperref}

\usepackage{bookmark}

\newcommand{\dual}[1]{#1^{*}}

\title{Anelli euclidei, PID e UFD}
\author{Gabriel Antonio Videtta}
\date{\today}

\DeclareMathOperator{\tr}{tr}
\DeclareMathOperator{\Ker}{Ker}
\DeclareMathOperator{\MCD}{MCD}
\DeclareMathOperator{\Imm}{Imm}

\setlength\parindent{0pt}

\let\oldforall\forall
\let\forall\undefined
\DeclareMathOperator{\forall}{\oldforall}

\let\oldexists\exists
\let\exists\undefined
\DeclareMathOperator{\exists}{\oldexists}

\let\oldnexists\nexists
\let\nexists\undefined
\DeclareMathOperator{\nexists}{\oldnexists}

\let\oldland\land
\let\land\undefined
\DeclareMathOperator{\land}{\oldland}

\let\oldlor\lor
\let\lor\undefined
\DeclareMathOperator{\lor}{\oldlor}

\let\oldlnot\lnot
\let\lnot\undefined
\DeclareMathOperator{\lnot}{\oldlnot}

\let\oldcirc\circ
\let\circ\undefined
\DeclareMathOperator{\circ}{\oldcirc}

\DeclareMathOperator{\existsone}{\exists !}

\let\oldemptyset\emptyset
\let\emptyset\varnothing

\begin{document}

\maketitle

\newcommand{\nsg}{\mathrel{\unlhd}}

\newcommand{\BB}{\mathcal{B}}
\newcommand{\FF}{\mathbb{F}_2}
\newcommand{\NN}{\mathbb{N}}
\newcommand{\ZZ}{\mathbb{Z}}
\newcommand{\RR}{\mathbb{R}}
\newcommand{\KK}{\mathbb{K}}
\newcommand{\LL}[2]{\mathcal{L} \left(#1, \, #2\right)}

\newcommand{\MM}[2]{\mathcal{M}_{#1 \times #2}\left(\KK\right)}
\newcommand{\M}[1]{\mathcal{M}_{#1}\left(\KK\right)}
\newcommand{\Mbb}[3]{\mathcal{M}^{#1}_{#2} \left( #3 \right)}
\newcommand{\Mb}[2]{\mathcal{M}^{#1}_{#2}}

\theoremstyle{definition}
\newtheorem{definition}{Definizione}[section]

\renewcommand{\vec}[1]{\underline{#1}}

\newtheorem{example}{Esempio}[section]
\newtheorem{exercise}{Esercizio}[section]
\newtheorem{lemma}{Lemma}[section]
\newtheorem{theorem}{Teorema}[section]
\newtheorem{proposition}{Proposizione}[section]
\newtheorem{corollary}{Corollario}[section]
\newtheorem*{note}{Osservazione}

\tableofcontents

\section{Anelli euclidei e prime proprietà}

Nel corso della storia della matematica, numerosi studiosi hanno tentato
di generalizzare -- o meglio, accomunare a più strutture algebriche -- il
concetto di divisione euclidea che era stato formulato per l'anello
dei numeri interi $\ZZ$ e, successivamente, per l'anello dei polinomi
$\KK[x]$. Lo sforzo di questi studiosi ad oggi è converso in un'unica
definizione, quella di anello euclideo, di seguito presentata.

\begin{definition}
    Un \textbf{anello euclideo} è un dominio d'integrità $D$\footnote{Difatti, nella
        letteratura inglese, si parla di \textit{Euclidean domain} piuttosto che di
        anello.} sul quale è
    definita una funzione $g$ detta \textbf{funzione grado} o \textit{norma}
    soddisfacente le seguenti proprietà:

    \begin{itemize}
        \item $g : D \setminus \{0\} \to \NN$,
        \item $\forall a$, $b \in D \setminus \{0\}$, $g(a) \leq g(ab)$,
        \item $\forall a \in D$, $b \in D \setminus \{0\}$, $\exists q$, $r \in D \mid
                  a=bq+r$ e $r=0 \,\lor\, g(r)<g(q)$.
    \end{itemize}
\end{definition}

Di seguito vengono presentate alcune definizioni, correlate
alle proprietà immediate di un anello euclideo.

\begin{definition}
    Dato un anello euclideo $E$, siano $a \in E$ e $b \in E \setminus \{0\}$. Si dice che
    $b \mid a$, ossia che $b$ \textit{divide} $a$, se $\exists c \in E \mid
        a=bc$.
\end{definition}

\begin{note}
    Si osserva che, per ogni anello euclideo $E$, qualsiasi $a \in E$ divide
    $0$. Infatti, $0 = a0$.
\end{note}

\begin{proposition}
    Dato un anello euclideo $E$, $a \mid b \,\land\, b \nmid a \implies g(a) < g(b)$.
\end{proposition}

\begin{proof}
    Poiché $b \nmid a$, esistono $q$, $r$ tali che $a = bq + r$, con
    $g(r) < g(b)$. Dal momento però che $a \mid b$, $\exists c \mid
        b = ac$. Pertanto $a = ac + r \implies r = a(1-c)$. Dacché $1-c \neq 0$ --
    altrimenti $r=0$, \Lightning{} --, così come $a \neq 0$, si deduce
    dalle proprietà della funzione grado che $g(a) \leq g(r)$.
    Combinando le due disuguaglianze, si ottiene la
    tesi: $g(a) < g(b)$.
\end{proof}

\begin{proposition}
    \label{prop:g1_minimo}
    $g(1)$ è il minimo di $\Imm g$, ossia il minimo grado assumibile
    da un elemento di un anello euclideo $E$.
\end{proposition}

\begin{proof}
    Sia $a \in E \setminus \{0\}$, allora, per le proprietà della funzione
    grado, $g(1) \leq g(1a) = g(a)$.
\end{proof}

\begin{theorem}
    Sia $a \in E \setminus \{0\}$, allora $a \in E^* \iff g(a) = g(1)$.
\end{theorem}

\begin{proof}
    Dividiamo la dimostrazione in due parti, ognuna corrispondente a una implicazione. \\

    ($\implies$) \;Sia $a \in E^*$, allora $\exists b \in E^*$ tale che $ab=1$. Poiché
    sia $a$ che $b$ sono diversi da $0$, dalle proprietà della funzione grado si
    desume che $g(a) \leq g(ab) = g(1)$. Poiché, dalla \textit{Proposizione \ref{prop:g1_minimo}},
    $g(1)$ è minimo, si conclude che $g(a) = g(1)$. \\

    ($\;\Longleftarrow\;$) \;Sia $a \in E \setminus \{0\}$ con $g(a) = g(1)$. Allora
    esistono $q$, $r$ tali che $1 = aq + r$. Vi sono due possibilità: che $r$ sia $0$, o
    che $g(r) < g(a)$. Tuttavia, poiché $g(a)=g(1)$, dalla \textit{Proposizione \ref{prop:g1_minimo}} si desume che $g(a)$ è minimo, e quindi che
    $r$ è nullo. Si conclude quindi che $aq = 1$, e dunque che $a \in E^*$.
\end{proof}

\section{PID e UFD}

\subsection{Irriducibili e prime definizioni}

Come accade nell'aritmetica dei numeri interi, anche in un dominio è possibile definire
una nozione di \textit{primo}. In un dominio possono essere tuttavia definiti due tipi di "primi",
gli elementi \textit{irriducibili} e gli elementi \textit{primi}.

\begin{definition}
    In un dominio $A$, si dice che $a \in A \setminus A^*$ è \textbf{irriducibile} se
    $\exists b$, $c \mid a=bc \implies b \in A^*$ o $c \in A^*$.
\end{definition}

\begin{note}
    Dalla definizione si escludono gli invertibili di $A$ per permettere
    di definire meglio il concetto di fattorizzazione in seguito. Infatti,
    se li avessimo inclusi, avremmo che ogni dominio sarebbe a fattorizzazione
    non unica, dal momento che $a=bc$ potrebbe essere scritto anche come
    $a=1bc$.
\end{note}

\begin{definition}
    In un dominio $A$, si dice che $a \in A \setminus A^*$ è \textbf{primo} se
    $a \mid bc \implies a \mid b \,\lor\, a \mid c$.
\end{definition}

\begin{proposition}
    Se $a \in A$ è primo, allora $a$ è anche irriducibile.
\end{proposition}

\begin{proof}
    Si dimostra la tesi contronominalmente. Sia $a$ non irriducibile. Se
    $a \in A^*$, allora $a$ non può essere primo. Altrimenti $a=bc$ con
    $b$, $c \in A \setminus A^*$. \\

    Chiaramente $a \mid bc$, ossia sé stesso. Senza perdità di generalità, se $a \mid b$, esiste $d \in A$ tale per cui
    $b=ad$. Pertanto, $a=adc \implies a(1-dc)=0$. Poiché $A$ è un dominio,
    uno dei due fattori deve essere nullo. $a$ non può esserlo, perché $0$ non
    può dividere $b$. Tuttavia neanche $dc-1$ può essere nullo, altrimenti
    si verificherebbe che $dc=1$, e quindi che $c \in A^*$, \Lightning{}.
\end{proof}

\begin{definition}
    Si dice che due elementi non nulli $a$, $b$ appartenenti a un anello euclideo
    $E$ sono \textbf{associati} se $a \mid b$ e $b \mid a$.
\end{definition}

\begin{proposition}
    \label{prop:associati}
    $a$ e $b$ sono associati $\iff \exists c \in E^* \mid a=bc$ e $a$, $b$ entrambi non nulli.
\end{proposition}

\begin{proof} Si dimostrano le due implicazioni separatamente. \\

    ($\implies$) Se $a$ e $b$ sono associati, allora $\exists d$, $e$ tali che $a=bd$ e che $b=ae$. Combinando le due relazioni si ottiene:

    \[ a=aed \implies a(1-ed)=0.\]

    Poiché $a$ è diverso da zero, si ricava che $ed=1$, ossia
    che $d$, $e \in E^*$, e quindi la tesi. \\

    ($\;\Longleftarrow\;$) Se $a$ e $b$ sono entrambi non
    nulli e $\exists c \in E^* \mid a=bc$, $b$ chiaramente
    divide $a$. Inoltre, $a=bc \implies b=ac^{-1}$, e quindi
    anche $a$ divide $b$. Pertanto $a$ e $b$ sono associati.
\end{proof}

\begin{proposition}
    \label{prop:divisione_associati}
    Siano $a$ e $b$ due associati in $E$. Allora $a \mid c \implies b \mid c$.
\end{proposition}

\begin{proof}
    Poiché $a$ e $b$ sono associati, per la \textit{Proposizione \ref{prop:associati}}, $\exists d \in E^*$ tale che
    $a = db$. Dal momento che $a \mid c$, $\exists \alpha \in E$ tale che
    $c = \alpha a$, quindi:

    \[ c = \alpha a = \alpha d b,\]

    da cui la tesi.
\end{proof}

\begin{proposition}
    \label{prop:associati_generatori}
    Siano $a$ e $b$ due associati in $E$. Allora
    $(a)=(b)$.
\end{proposition}

\begin{proof}
    Poiché $a$ e $b$ sono associati, $\exists d \in E^*$ tale che $a = db$. Si dimostra l'uguaglianza dei due insiemi. \\

    Sia $\alpha = ak \in (a)$, allora $\alpha = dbk$ appartiene anche a $(b)$, quindi $(a) \subseteq (b)$. Sia
    invece $\beta = bk \in (b)$, allora $\beta = d^{-1}ak$
    appartiene anche a $(a)$, da cui $(b) \subseteq (a)$.
    Dalla doppia inclusione si verifica la tesi, $(a)=(b)$.
\end{proof}

\subsection{PID e MCD}

Come accade per $\ZZ$, in ogni anello euclideo è possibile definire il
concetto di \textit{massimo comun divisore}, sebbene con qualche accortezza
in più. Pertanto, ancor prima di definirlo, si enuncia la definizione di
PID e si dimostra un teorema fondamentale degli anelli euclidei, che
si ripresenterà in seguito come ingrediente fondamentale per la fondazione
del concetto di MCD.

\begin{definition}
    Si dice che un dominio è un \textit{principal ideal domain} (\textbf{PID})\footnote{Ossia un \textit{dominio
            a soli ideali principali}, quindi monogenerati, proprio come da definizione.} se ogni suo ideale è monogenerato.
\end{definition}

\begin{theorem}
    Sia $E$ un anello euclideo. Allora $E$ è un PID.
\end{theorem}

\begin{proof}
    Sia $I$ un ideale di $E$. Se $I = (0)$, allora $I$ è già monogenerato.
    Altrimenti si consideri l'insieme $g(I \setminus \{0\})$. Poiché
    $g(I \setminus \{0\}) \subseteq \NN$,
    esso ammette un minimo per il principio del buon ordinamento. \\

    Sia $m \in I$ un valore che assume tale minimo e sia $a \in I$.
    Poiché $E$ è euclideo, $\exists q$, $r \mid a = mq + r$ con $r=0$ o
    $g(r)<g(m)$. Tuttavia, poiché $r = a-mg \in I$ e $g(m)$ è minimo, necessariamente $r=0$ -- altrimenti $r$ sarebbe
    ancor più minimo di $m$, \Lightning{} --,
    quindi $m \mid a$, $\forall a \in I$. Quindi $I \subseteq (m)$. \\

    Dal momento che per le proprietà degli ideali $\forall a \in E$, $ma \in I$,
    si conclude che $(m) \subseteq I$. Quindi $I = (m)$.
\end{proof}

Adesso è possibile definire il concetto di massimo comun divisore, basandoci
sul fatto che ogni anello euclideo è un PID.

\begin{definition}
    Sia $D$ un dominio e siano $a$, $b \in D$. Si definisce
    \textit{massimo comun divisore} (\textbf{MCD}) di $a$ e $b$ un
    generatore dell'ideale $(a,b)$.
\end{definition}

\begin{note}
    Questa definizione di MCD è una buona definizione dal momento che sicuramente
    esiste un generatore dell'ideale $(a,b)$, dacché $D$ è un PID.
\end{note}

\begin{note}
    Non si parla di un unico massimo comun divisore, dal momento che
    potrebbero esservi più generatori dell'ideale $(a,b)$. Segue tuttavia che tutti questi generatori sono in realtà
    associati\footnote{Infatti ogni generatore divide ogni
        altro elemento di un ideale, e così i vari generatori si
        dividono tra di loro. Pertanto sono associati.}.
    Quando si scriverà
    $\MCD(a,b)$ s'intenderà quindi uno qualsiasi di questi associati.
\end{note}

\begin{theorem}[\textit{Identità di Bézout}]
    \label{th:bezout}
    Sia $d$ un MCD di $a$ e $b$. Allora
    $\exists \alpha$, $\beta$ tali che $d = \alpha a + \beta b$.
\end{theorem}

\begin{proof}
    Il teorema segue dalla definizione di MCD come generatore
    dell'ideale $(a,b)$. Infatti, poiché $d \in (a,b)$, esistono
    sicuramente, per definizione, $\alpha$ e $\beta$ tali che
    $d = \alpha a + \beta b$.
\end{proof}

\begin{proposition}
    \label{prop:mcd}
    Siano $a$, $b \in D$. Allora vale la seguente equivalenza:

    \[ d = \MCD(a, b) \iff \begin{cases} d \mid a \,\land\, d \mid b \\ \forall c \text{ t.c.\ } c \mid a \,\land\, c \mid b,\;c \mid d\end{cases}\]
\end{proposition}

\begin{proof} Si dimostrano le due implicazioni separatamente. \\

    ($\implies$) Poiché $d$ è generatore dell'ideale $(a, b)$, la prima proprietà segue banalmente. \\

    Inoltre, per l'\nameref{th:bezout}, $\exists \alpha$, $\beta$ tali che
    $d = \alpha a + \beta b$. Allora, se $c \mid a$ e $c \mid b$, sicuramente
    esistono $\gamma$ e $\delta$ tali che $a=\gamma c$ e $b=\delta c$. Pertanto
    si verifica la seconda proprietà, e quindi la tesi:

    \[ d = \alpha a + \beta b = \alpha \gamma c + \beta \delta c = c(\alpha\gamma+\beta\delta). \]

    \vskip 0.1in

    ($\;\Longleftarrow\;$) Sia $m = \MCD(a,b)$. Dal momento che $d$ divide
    sia $a$ che $b$, $d$ deve dividere, per l'implicazione scorsa, anche $m$.
    Per la seconda proprietà, $m$ divide $d$ a sua volta. Allora $d$ è un
    associato di $m$, e quindi, dalla \textit{Proposizione \ref{prop:associati_generatori}}, $(m)=(d)=(a,b)$, da cui $d = \MCD(a,b)$.
\end{proof}

\begin{proposition}
    \label{prop:divisione_gcd}
    Se $a \mid bc$ e $d = \MCD(a, b) \in D^*$, allora $a \mid c$.
\end{proposition}

\begin{proof}
    Per l'\nameref{th:bezout} $\exists \alpha$, $\beta$ tali che
    $\alpha a + \beta b = d$. Allora, poiché $a \mid bc$, $\exists
        \gamma$ tale che $bc=a\gamma$. Si verifica quindi la tesi:

    \[ \alpha a + \beta b = d \implies \alpha ac + \beta bc = dc \implies
        a d^{-1} (\alpha c + \beta \gamma) = c.\]
\end{proof}

\begin{lemma}
    \label{lem:primalità_mcd}
    Se $a$ è un irriducibile di un PID $D$, allora $\forall b \in D$,
    $(a,b)=D \,\lor\, (a,b)=(a)$, o equivalentemente $\MCD(a,b) \in D^*$ o
    $\MCD(a,b) = a$.
\end{lemma}

\begin{proof}
    Dacché $\MCD(a,b) \mid a$, le uniche opzioni, dal momento che $a$ è irriducibile,
    sono che $\MCD(a,b)$ sia un invertibile o che sia un associato
    di $a$ stesso.
\end{proof}

\begin{theorem}
    \label{th:irriducibili_primi}
    Se $a$ è un irriducibile di un PID $D$, allora $a$ è anche un primo.
\end{theorem}

\begin{proof}
    Siano $b$ e $c$ tali che $a \mid bc$. Per il \textit{Lemma \ref{lem:primalità_mcd}},
    $\MCD(a,b)$ può essere solo un associato di $a$ o essere un invertibile. Se è
    un associato di $a$, allora, per la \textit{Proposizione \ref{prop:divisione_associati}}, poiché $\MCD(a,b)$ divide $b$, anche $a$ divide $b$.
    Altrimenti $\MCD(a,b) \in D^*$, e quindi, per la \textit{Proposizione \ref{prop:divisione_gcd}}, $a \mid c$.
\end{proof}

\subsection{L'algoritmo di Euclide}

Per algoritmo di Euclide si intende un algoritmo che è in grado di
produrre in un numero finito di passi un MCD tra due elementi
$a$ e $b$ non entrambi nulli di un anello euclideo\footnote{Si richiede che l'anello sia
    euclideo e non soltanto che sia un PID, dal momento che l'algoritmo
    usufruisce delle proprietà della funzione grado.}. L'algoritmo
classico è di seguito presentato:

\newpage

\begin{algorithm}
    $e \gets \max(a,b)$\;
    $d \gets \min(a,b)$\;
    \BlankLine\BlankLine
    \While{$d>0$}
    {
        $m \gets d$\;
        $d \gets e \bmod d$\;
        $e \gets m$\;
    }
\end{algorithm}

dove $e$ è l'MCD ricercato e l'operazione $\mathrm{mod}$ restituisce un resto della
divisione euclidea\footnote{Ossia $a \bmod b$ restituisce un $r$ tale che $\exists q
        \mid a = bq+r$ con $r=0$ o $g(r)<g(q)$.}.

\begin{lemma}
    \label{lem:euclide_finito}
    L'algoritmo di Euclide termina sempre in un numero finito di passi.
\end{lemma}

\begin{proof}
    Se $d$ è pari a $0$, l'algoritmo termina immediatamente. \\

    Altrimenti si può costruire una sequenza $(g(d_i))_{i\geq1}$ dove $d_i$ è il valore di $d$ all'inizio
    di ogni $i$-esimo ciclo $\textbf{while}$. Ad ogni ciclo vi sono due casi: se $d_i$ si annulla dopo
    l'operazione di $\mathrm{mod}$, il ciclo si conclude al passo successivo, altrimenti,
    poiché $d_i$ è un resto di una divisione euclidea, segue che $g(d_i)<g(d_{i-1})$, dove
    si pone $d_{0}=\min(a, b)$. \\

    Per il principio della discesa infinita, $(g(d_i))_{i\geq1}$ non può essere
    una sequenza infinita, essendo strettamente decrescente. Quindi la sequenza è
    finita, e pertanto il ciclo $\textbf{while}$ s'interrompe dopo un numero finito
    di passi.
\end{proof}

\begin{lemma}
    \label{lem:generatori_euclide}
    Sia $r = a \bmod b$. Allora vale che $(a,b)=(b,r)$.
\end{lemma}

\begin{proof}
    Poiché $r = a \bmod b$, $\exists q$ tale che $a = qb + r$.
    Siano $k_1$ e $k_2$ tali che $(k_1)=(a,b)$ e $(k_2)=(b,r)$. Dal
    momento che $k_1$ divide sia $a$ che $b$, si ha che divide anche
    $r$. Siano $\alpha$, $\beta$ tali che $a = \alpha k_1$ e
    $b = \beta k_1$. Si verifica infatti che:

    \[ r = a - qb = \alpha k_1 - q \beta k_1 = k_1 (\alpha - q \beta). \]

    Poiché $k_1$ divide sia $b$ che $r$, per le proprietà del $\MCD$,
    $k_1$ divide anche $k_2$. Analogamente, $k_2$ divide $k_1$. Pertanto
    $k_1$ e $k_2$ sono associati, e dalla \textit{Proposizione \ref{prop:associati_generatori}} generano quindi lo stesso ideale, da
    cui la tesi.
\end{proof}

\begin{theorem}
    L'algoritmo di Euclide restituisce sempre correttamente un MCD tra due elementi $a$ e $b$ non entrambi nulli in un numero finito di passi.
\end{theorem}

\begin{proof}
    Per il \textit{Lemma \ref{lem:euclide_finito}}, l'algoritmo sicuramente termina.
    Se $d$ è pari a $0$, allora l'algoritmo termina restituendo $e$. Il valore è
    corretto, dal momento che, senza perdità di generalità, se $b$ è nullo, allora
    $\MCD(a, b)=a$: infatti $a$ divide sia sé stesso che $0$, e ogni divisore di $a$ è
    sempre un divisore di $0$. \\

    Se invece $d$ non è pari a $0$, si scelga il $d_n$ tale che $g(d_n)$ sia l'ultimo
    elemento della sequenza $(g(d_i))_{i\geq1}$ definita nel \textit{Lemma \ref{lem:euclide_finito}}. Per il \textit{Lemma \ref{lem:generatori_euclide}},
    si ha la seguente uguaglianza:

    \[ (e_0, d_0) = (d_0, d_1) = \cdots = (d_n, 0) = (d_n). \]

    \vskip 0.1in

    Poiché quindi $d_n$ è generatore di $(e_0, d_0)=(a,b)$, $d_n = \MCD(a,b)$.
\end{proof}

\subsection{UFD e fattorizzazione}

Si enuncia ora la definizione fondamentale di UFD, sulla
quale costruiremo un teorema fondamentale per gli anelli
euclidei.

\begin{definition}
    Si dice che un dominio $D$ è uno \textit{unique factorization domain} (\textbf{UFD})\footnote{Ossia
        un \textit{dominio a fattorizzazione unica}.} se ogni $a \in D$ non nullo e non invertibile può essere scritto
    in forma unica come prodotto di irriducibili, a meno di associati.
\end{definition}

\begin{lemma}
    \label{lem:fattorizzazione}
    Sia $E$ un anello euclideo. Allora ogni elemento $a \in E$ non nullo e
    non invertibile può essere scritto come prodotto di irriducibili.
\end{lemma}

\begin{proof}
    Si definisca $A$ nel seguente modo:

    \[A = \{g(a) \mid a \in E \setminus (E^* \cup \{0\}) \text{ non sia prodotto di irriducibili}\}.\]

    \vskip 0.1in

    Se $A \neq \emptyset$, allora, poiché $A \subseteq \NN$, per il principio
    del buon ordinamento, esiste un $m \in E$ tale che $g(m)$ sia minimo.
    Sicuramente $m$ non è irriducibile -- altrimenti $g(m) \notin A$, \Lightning{} --,
    quindi $m=ab$ con $a$, $b \in E \setminus E^*$. \\

    Poiché $a \mid m$, ma $m \nmid a$ -- altrimenti $a$ e $m$ sarebbero
    associati, e quindi $b$ sarebbe invertibile --, si deduce che $g(a) < g(m)$, e
    quindi che $g(a) \notin A$. Allora $a$ può scriversi come prodotto di irriducibili.
    Analogamente anche $b$ può scriversi come prodotto di irriducibili, e quindi
    $m$, che è il prodotto di $a$ e $b$, è prodotto di irriducibili, \Lightning{}. \\

    Quindi $A = \emptyset$, e ogni $a \in E$ non nullo e non invertibile è prodotto
    di irriducibili.
\end{proof}

\begin{theorem}
    \label{th:euclidei_ufd}
    Sia $E$ un anello euclideo. Allora $E$ è un UFD\footnote{In realtà questo teorema
        è un caso particolare di un teorema più generale: ogni PID è un UFD. Poiché
        la dimostrazione esula dalle intenzioni di questo articolo, si è preferito
        dimostrare il caso più familiare. Per la dimostrazione del teorema più generale si
        rimanda a \cite[pp.~124-126]{di2013algebra}.}.
\end{theorem}

\begin{proof}
    Innanzitutto, per il \textit{Lemma \ref{lem:fattorizzazione}}, ogni
    $a \in E$ non invertibile e non nullo ammette una fattorizzazione. \\

    Sia allora $a \in E$ non invertibile e non nullo. Affinché $E$ sia un UFD,
    deve verificarsi la seguente condizione: se
    $a=p_1p_2 \cdots p_r=q_1q_2 \cdots q_s \in E$, allora
    $r=s$ ed esiste una permutazione $\sigma \in S_r$ tale per cui
    $\sigma$ associ a ogni indice $i$ di un $p_i$ un indice $j$ di
    un $q_j$ in modo tale che $p_i$ e $q_j$ siano associati. \\

    Si procede per induzione. \\

    (\textit{passo base}) \,Se $r=1$, allora $a$ è irriducibile. Allora necessariamente
    $s=1$, altrimenti $a$ sarebbe prodotto di irriducibili, e quindi contemporaneamente
    anche non irriducibile. Inoltre esiste la permutazione banale $e \in S_1$ che
    associa $p_1$ a $q_1$. \\

    (\textit{passo induttivo}) \,Si assume che valga la tesi se $a$ è
    prodotto di $r-1$ irriducibili.
    Si consideri $p_1$: poiché $p_1$ divide $a$, $p_1$ divide anche
    $q_1q_2 \cdots q_s$. Dal momento che $E$, in quanto
    anello euclideo, è anche un dominio, dal \textit{Teorema \ref{th:irriducibili_primi}}, $p_1$ è anche primo,
    e quindi $p_1 \mid q_1$ o $p_1 \mid q_2 \cdots q_s$. \\

    Se $p_1 \nmid q_1$ si reitera il procedimento su $q_2 \cdots q_s$, trovando in
    un numero finito di passi un $q_j$ tale per cui $p_1 \mid q_j$. Allora si procede
    la dimostrazione scambiando $q_1$ e $q_j$. \\

    Poiché $q_1$ è irriducibile, $p_1$ e $q_1$ sono associati, ossia $q_1 = kp_1$ con
    $k \in E^*$. Allora $p_1 \cdots p_r = q_1 \cdots q_s = kp_1 \cdots q_s$, quindi,
    dal momento che $p_1 \neq 0$ ed $E$ è un dominio:

    \[p_1(p_2 \cdots p_r - kq_2 \cdots q_s)=0 \implies p_2 \cdots p_r = kq_2 \cdots q_s .\]

    Tuttavia il primo membro è un prodotto $r-1$ irriducibili, pertanto $r=s$ ed
    esiste un $\sigma \in S_{r-1}$ che associa ad ogni irriducibile $p_i$ un suo
    associato $q_i$. Allora si estende $\sigma$ a $S_r$ mappando $p_1$ a $q_1$,
    verificando la tesi.
\end{proof}

\section{Esempi notevoli di anelli euclidei}

\subsection{I numeri interi: $\ZZ$}

Senza ombra di dubbio l'esempio più importante di anello euclideo -- nonché
l'esempio da cui si è generalizzata proprio la stessa nozione di anello
euclideo -- è l'anello dei numeri interi. \\

In questo dominio la funzione grado è canonicamente il valore assoluto:

\[g : \ZZ \setminus \{0\} \to \NN, \, k \mapsto \left|k\right|.\]

\vskip 0.1in

Infatti, chiaramente $|a| \leq |ab|\, \forall a$, $b \in \ZZ \setminus \{0\}$. Inoltre
esistono -- e sono anche unici, a meno di segno -- $q$, $r \in \ZZ \mid a = bq + r$, con $r=0 \,\lor\,
    \left|r\right| < \left|q\right|$. \\

Dal momento che così si verifica che $\ZZ$ è un anello euclideo, il \textit{Teorema
    fondamentale dell'aritmetica} è una conseguenza del
\textit{Teorema \ref{th:euclidei_ufd}}.

\subsection{I campi: $\KK$}

Ogni campo $\KK$ è un anello euclideo, seppur banalmente. Infatti, eccetto proprio
per $0$, ogni elemento è "divisibile" per ogni altro elemento: siano $a$, $b \in \KK$,
allora $a = ab^{-1}b$. \\

Si definisce quindi la funzione grado come la funzione nulla:

\[g : \KK^* \to \NN, \, a \mapsto 0.\]

\vskip 0.1in

Chiaramente $g$ soddisfa il primo assioma della funzione grado. Inoltre,
poiché ogni elemento è "divisibile", il resto è sempre zero -- non è pertanto
necessario verificare nessun'altra proprietà.

\subsection{I polinomi di un campo: $\KK[x]$}

I polinomi di un campo $\KK$ formano un anello euclideo rilevante
nello studio dell'algebra astratta. Come suggerisce la
terminologia, la funzione grado in questo dominio coincide
proprio con il grado del polinomio, ossia si definisce come:

\[g : \KK[x] \setminus \{0\} \to \NN, \, f(x) \mapsto \deg f.\]

\vskip 0.1in

Si verifica facilmente che $g(a(x)) \leq g(a(x)b(x)) \, \forall a(x)$, $b(x) \in \KK[x] \setminus \{0\}$, mentre la divisione euclidea -- come negli interi -- ci permette
di concludere che effettivamente $\KK[x]$ soddisfa tutti gli assiomi di un anello
euclideo\footnote{Curiosamente i polinomi di $\KK[x]$ e i campi $\KK$ sono gli unici anelli euclidei in cui resti
    e quozienti sono unici, includendo la scelta di segno (vd.
    \cite{10.2307/2315810}).}.

\begin{example}
    Sia $\alpha \in \KK$ e sia $\varphi_\alpha : \KK[x] \to \KK, \, f(x) \mapsto f(\alpha)$
    la sua valutazione polinomiale in $\KK[x]$. $\varphi_\alpha$ è un omomorfismo, il cui
    nucleo è rappresentato dai polinomi in $\KK[x]$ che hanno $\alpha$ come radice. Poiché
    $\KK[x]$ è un PID, $\Ker \varphi$ deve essere monogenerato. $x-\alpha \in \Ker \varphi$
    è irriducibile, e quindi è il generatore dell'ideale. Si desume così che
    $\Ker \varphi = (x-\alpha)$.
\end{example}

\subsection{Gli interi di Gauss: $\ZZ[i]$}

Un importante esempio di anello euclideo è il dominio degli interi di Gauss $\ZZ[i]$, definito come:

\[\ZZ[i] = \{a+bi \mid a, b \in \ZZ\}.\]

\vskip 0.1in

\begin{wrapfigure}{l}{0pt}
    \begin{tikzpicture}
        \begin{scope}
            \clip (-2, -0.5) rectangle (3, 3);
            \draw[step=0.25cm, gray!20!white, very thin] (-7, -3) grid (7, 3);

            \foreach \x in {-4,...,4} {
                    \draw[ultra thin, loosely dashdotted] (-3 + \x, -3) -- (3 + \x, 3);
                }
            \foreach \y in {-4,...,5} {
                    \draw[ultra thin, loosely dashdotted] (-7, 7 + \y) -- (7, -7 + \y);
                }

            \draw[line width=0.7pt, ->] (0, 0) -- (0.5, 0.5) node[align=center, below=3pt]{$b$};
            \draw[line width=0.7pt, ->] (0, 0) -- (-0.5, 0.5) node[align=center, below=2pt]{$ib$};

            \draw[line width=0.5pt, ->] (0, 0) -- (0.5, 2.5) node[above=0.5pt]{$bq$};

            \draw[line width=0.5pt, ->] (0, 0) -- (1, 2.5) node[below, right]{$a$};

            \draw[densely dotted] (0.5, 2.5) -- (1, 2.5) node[below=4pt, left=2.5pt]{$r$};

            \draw[line width=0.2pt, ->] (0, -1) -- (0, 3);
            \draw[line width=0.2pt, ->] (-3, 0) -- (3, 0);

        \end{scope}
    \end{tikzpicture}
    \caption{Visualizzazione della divisione euclidea nel piano degli interi di Gauss.}
    \label{fig:z_i}
\end{wrapfigure}

La funzione grado coincide in particolare con il quadrato del modulo di un numero complesso, ossia:
\[g(z) : \ZZ[i] \setminus \{0\} \to \NN, \, a+bi \mapsto \left| a+bi \right|^2.\]

Il vantaggio di quest'ultima definizione è l'enfasi sul collegamento tra la funzione grado
di $\ZZ$ e quella di $\ZZ[i].$ Infatti, se $a \in \ZZ$, il grado di $a$ in $\ZZ$ e in $\ZZ[i]$
sono uno il quadrato dell'altro. In particolare, è possibile ridefinire il grado
di $\ZZ$ proprio in modo tale da farlo coincidere con quello di $\ZZ[i]$. \\

\newpage

\begin{theorem}
    $\ZZ[i]$ è un anello euclideo.
\end{theorem}

\begin{proof}
    Si verifica la prima proprietà della funzione grado. Siano $a$, $b \in \ZZ[i] \setminus \{0\}$,
    allora $\left|a\right| \geq 1 \,\land\, \left|b\right| \geq 1$. Poiché
    $\left|ab\right| = \left|a\right|\left|b\right|$\footnote{Questa interessante proprietà del modulo è alla base dell'identità di Brahmagupta-Fibonacci: $(a^2 + b^2)(c^2 + d^2) = (ac-bd)^2 + (ad+bc)^2.$}, si verifica facilmente che
    $\left|ab\right| \geq \left|a\right|$, ossia che $g(ab) \geq g(a)$. \\

    Si verifica infine che esiste una divisione euclidea, ossia che
    $\forall a \in \ZZ[i]$, $\forall b \in \ZZ[i] \setminus \{0\}$, $\exists q$, $r \in \ZZ[i] \mid a = bq + r$ e $r=0 \,\lor\, g(r) < g(b)$.
    Come si visualizza facilmente nella \textit{Figura \ref{fig:z_i}},
    tutti i multipli di $b$ formano un piano con basi $b$ e $ib$, dove
    sicuramente esiste un certo $q$ tale che la distanza $\left|r\right| = \left|a-bq\right|$ sia minima. \\

    Se $a$ è un multiplo di $b$, vale sicuramente che $a = bq$. Altrimenti dal momento che $r$ è sicuramente inquadrato in uno dei tasselli del piano, vale
    sicuramente la seguente disuguaglianza, che lega il modulo di $r$ alla diagonale di
    ogni quadrato:

    \[\left|r\right| \leq \frac{\left|b\right|}{\sqrt{2}}.\]

    Pertanto vale la seconda e ultima proprietà della funzione grado:

    \[\left|r\right| \leq \frac{\left|b\right|}{\sqrt{2}} < \left|b\right| \implies \left|r\right|^2 < \left|b\right|^2 \implies g(r) < g(b).\]
\end{proof}

\subsection{Gli interi di Eisenstein: $\ZZ[\omega]$}

Sulla scia di $\ZZ[i]$ è possibile definire anche l'anello degli
interi di Eisenstein, aggiungendo a $\ZZ$ la prima radice cubica
primitiva dell'unità in senso antiorario, ossia:

\[\omega = e^{\frac{2\pi i}{3}} = -\frac{1}{2} + \frac{\sqrt{3}}{2}i.\]

In particolare, $\omega$ è una delle due radici dell'equazione
$z^2 + z + 1 = 0$, dove invece l'altra radice altro non è che
$\omega^2 = \overline{\omega}$.

\begin{wrapfigure}{l}{0pt}
    \begin{tikzpicture}
        \begin{scope}
            \clip (-2, -0.5) rectangle (3, 3);
            \draw[step=0.25cm, gray!20!white, very thin] (-7, -3) grid (7, 3);

            \foreach \x in {-4,...,4} {
                    \draw[ultra thin, loosely dashdotted] (-3 + 0.87*\x, -3) -- (3 + 0.87*\x, 3);
                }

            \foreach \y in {-4,...,5} {
                    \draw[ultra thin, loosely dashdotted] (-7, 1.8756443470179 + 0.65*\y) -- (7, -1.8756443470179 + 0.65*\y);
                }

            \foreach \x in {-4,...,5} {
                    \draw[ultra thin, loosely dashed] (-7 + 0.6289*\x, 28.5025773880714) -- (7+ 0.65*\x, -28.5025773880714);
                }

            \draw[line width=0.7pt, ->] (0, 0) -- (0.5, 0.5) node[align=center, below=3pt]{$b$};
            \draw[line width=0.7pt, ->] (0, 0) -- (-0.6830127018922, 0.1830127018922) node[align=center, below=2pt]{$\omega b$};

            \draw[line width=0.5pt, ->] (0, 0) -- (0.71494, 2.41094) node[below=2pt, left=4pt]{$bq$};

            \draw[line width=0.5pt, ->] (0, 0) -- (1.1, 2.7) node[below, right]{$a$};

            \draw[densely dotted] (0.71494, 2.41094) -- (1.1, 2.7) node[above=3pt, left=2.5pt]{$r$};

            \draw[line width=0.2pt, ->] (0, -1) -- (0, 3);
            \draw[line width=0.2pt, ->] (-3, 0) -- (3, 0);

        \end{scope}
    \end{tikzpicture}
    \caption{Visualizzazione della divisione euclidea nel piano degli interi di Eisenstein.}
    \label{fig:z_omega}
\end{wrapfigure}

\vskip 0.1in

La funzione grado in $\ZZ[\omega]$ deriva da quella di $\ZZ[i]$ e coincide ancora
con il quadrato del modulo del numero complesso. Si definisce quindi:

\[g : \ZZ[\omega] \setminus \{0\}, \, a+b\omega \mapsto \left|a+b\omega\right|^2.\]

Sviluppando il modulo è possibile ottenere una formula più concreta:

\[ \left|a+b\omega\right|^2 = \left|\left(a-\frac{b}{2}\right) + \frac{b\sqrt{3}}{2}i\right|^2 =\] \\

\[= \left(a-\frac{b}{2}\right)^2 + \frac{3b^2}{4} = a^2 - ab + b^2.\] \\

\begin{theorem}
    $\ZZ[\omega]$ è un anello euclideo.
\end{theorem}

\begin{proof}
    Sulla scia della dimostrazione presentata per $\ZZ[i]$, si verifica facilmente
    la prima proprietà della funzione grado. Siano $a$, $b \in \ZZ[\omega]$, allora
    $\left|a\right| \geq 1$ e $\left|b\right| \geq 1$. Poiché dalle proprietà
    dei numeri complessi vale ancora $\left|a\right| \left|b\right| \geq \left|a\right|$,
    la proprietà $g(ab) \geq g(a)$ è già verificata. \\

    Si verifica infine la seconda e ultima proprietà della funzione grado. Come per
    $\ZZ[i]$, i multipli di $b \in \ZZ[\omega]$ sono visualizzati su un piano che
    ha per basi $b$ e $\omega b$ (come in $\textit{Figura \ref{fig:z_omega}}$), pertanto
    esiste sicuramente un $q$ tale che la distanza $\left|a-bq\right|$ sia minima. \\

    Se $a$ è multiplo di $b$, allora chiaramente $a = bq$. Altrimenti, $a$ è certamente
    inquadrato in uno dei triangoli del piano, per cui vale la seguente disuguaglianza:

    \[\left|r\right| \leq \frac{\sqrt{3}}{2} \left|b\right|.\]

    Dunque la tesi è verificata:

    \[\left|r\right| \leq \frac{\sqrt{3}}{2} \left|b\right| < \left|b\right| \implies \left|r\right|^2 < \left|b\right|^2 \implies g(r) < g(b). \]
\end{proof}

\section{Riferimenti bibliografici}

\printbibliography[heading=none]

\end{document}
