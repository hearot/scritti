\documentclass[a4paper]{article}
\usepackage[utf8]{inputenc}
\usepackage[italian]{babel}
\usepackage{algorithm2e}
\usepackage{amsfonts}
\usepackage{amsthm}
\usepackage{amssymb}
\usepackage{amsopn}
\usepackage{bm}
\usepackage{csquotes}
\usepackage{enumerate}
\usepackage{mathtools}
\usepackage{marvosym}
\usepackage{tikz}
\usepackage{wrapfig}
\usepackage{xpatch}

\usepackage[colorlinks=false,bookmarksopen=true]{hyperref}
\usepackage{bookmark}

\newcommand{\norm}[1]{\left|#1\right|}

\newcommand{\zeroset}{\{0\}}
\newcommand{\setminuszero}{\setminus \{0\}}

\newcommand{\BB}{\mathcal{B}}
\newcommand{\CC}{\mathbb{C}}
\newcommand{\FF}{\mathbb{F}_2}
\newcommand{\HH}{\mathbb{H}}
\newcommand{\KK}{\mathbb{K}}
\newcommand{\NN}{\mathbb{N}}
\newcommand{\QQ}{\mathbb{Q}}
\newcommand{\RR}{\mathbb{R}}
\newcommand{\ZZ}{\mathbb{Z}}
\newcommand{\ZZp}{\mathbb{Z}_p}

\newcommand{\CCx}{\mathbb{C}[x]}
\newcommand{\QQx}{\mathbb{Q}[x]}
\newcommand{\RRx}{\mathbb{R}[x]}
\newcommand{\ZZi}{\mathbb{Z}[i]}
\newcommand{\ZZom}{\mathbb{Z}[\omega]}
\newcommand{\ZZpx}{\mathbb{Z}_p[x]}
\newcommand{\ZZx}{\mathbb{Z}[x]}

\newcommand{\dual}[1]{#1^{*}}
\newcommand{\LL}[2]{\mathcal{L} \left(#1, \, #2\right)}
\newcommand{\M}[1]{\mathcal{M}_{#1}\left(\KK\right)}
\newcommand{\nsg}{\mathrel{\unlhd}}
\renewcommand{\vec}[1]{\underline{#1}}

\newcommand{\hatpi}{\hat{\pi}}
\newcommand{\hatpip}{\hat{\pi}_p}

\theoremstyle{definition}

\newtheorem{corollary}{Corollario}[section]
\newtheorem{definition}{Definizione}[section]
\newtheorem{example}{Esempio}[section]
\newtheorem{exercise}{Esercizio}[section]
\newtheorem{lemma}{Lemma}[section]
\newtheorem*{note}{Osservazione}
\newtheorem{proposition}{Proposizione}[section]
\newtheorem{theorem}{Teorema}[section]

\DeclareMathOperator{\existsone}{\exists !}
\DeclareMathOperator{\Imm}{Imm}
\DeclareMathOperator{\Ker}{Ker}
\DeclareMathOperator{\MCD}{MCD}
\DeclareMathOperator{\mcm}{mcm}
\DeclareMathOperator{\tr}{tr}

\let\oldemptyset\emptyset
\let\emptyset\varnothing

\let\oldcirc\circ
\let\circ\undefined
\DeclareMathOperator{\circ}{\oldcirc}

\let\oldexists\exists
\let\exists\undefined
\DeclareMathOperator{\exists}{\oldexists}

\let\oldforall\forall
\let\forall\undefined
\DeclareMathOperator{\forall}{\oldforall}

\let\oldnexists\nexists
\let\nexists\undefined
\DeclareMathOperator{\nexists}{\oldnexists}

\let\oldland\land
\let\land\undefined
\DeclareMathOperator{\land}{\oldland}

\let\oldlnot\lnot
\let\lnot\undefined
\DeclareMathOperator{\lnot}{\oldlnot}

\let\oldlor\lor
\let\lor\undefined
\DeclareMathOperator{\lor}{\oldlor}

\setlength\parindent{0pt}

\title{Proprietà fondamentali di $\ZZi$, \\ $\ZZx$, $\ZZpx$ e $\QQx$}
\author{Gabriel Antonio Videtta}
\date{\today}

\begin{document}

\maketitle

\tableofcontents

\section{Irriducibili e corollari di aritmetica in $\ZZi$}

Come già dimostrato, $\ZZi$ è un anello euclideo con la seguente
funzione grado:

\[ g : \ZZi \setminuszero \to \ZZ,\, a+bi \mapsto  \norm{a+bi}^2.\]

A partire da questo preconcetto è possibile dimostrare un teorema
importante in aritmetica, il \nameref{th:teorema_natale},
che discende direttamente come corollario di un teorema più
generale riguardante $\ZZi$.

\subsection{Il teorema di Natale di Fermat e gli irriducibili in $\ZZi$}

\begin{lemma}
    \label{lem:riducibile_due_quadrati}
    Sia $p$ un numero primo riducibile in $\ZZi$, allora $p$
    può essere scritto come somma di due quadrati in $\ZZ$.
\end{lemma}

\begin{proof}
    Se $p$ è riducibile in $\ZZi$, allora esistono $a+bi$ e
    $c+di$ appartenenti a $\ZZi \setminus \ZZi^*$ tali che $p=(a+bi)(c+di)$. \\

    Impiegando le proprietà dell'operazione di coniugio si
    ottiene la seguente equazione:

    \[ \overline{p}=p=(a-bi)(c-di) \implies p^2=p \overline{p} = (a^2+b^2)(c^2+d^2). \]

    Dal momento che $a+bi$ e $c+di$ non sono invertibili,
    i valori della funzione grado calcolati in essi sono strettamente
    maggiori del valore assunto nell'unità, ovverosia:

    \[ a^2+b^2>1, \qquad c^2+d^2>1. \]

    Allora devono per forza valere le seguenti equazioni:

    \[ p=a^2+b^2, \qquad p=c^2+d^2, \]

    da cui la tesi.
\end{proof}

\begin{lemma}
    \label{lem:quadrato_mod_4}
    Sia $p$ un numero primo tale che $p \equiv 1 \pmod4$. Allora
    esiste un $x \in \ZZ$ tale che $p \mid x^2+1$.
\end{lemma}

\begin{proof}
    Per il \textit{Teorema di Wilson}, $(p-1)! \equiv -1 \pmod p$.
    Attraverso varie manipolazioni algebriche si ottiene:

    \[-1 \equiv 1 \cdots \frac{p-1}{2} \cdot \frac{p+1}{2} \cdots (p-1) \equiv 1 \cdots \frac{p-1}{2} \left(-\frac{p-1}{2}\right) \cdots (-1) \equiv\]

    \[ \equiv (-1)^{\frac{p-1}{2}} \left(\left( \frac{p-1}{2} \right)!\right)^2 \equiv
        \left(\left( \frac{p-1}{2} \right)!\right)^2 \pmod p,
    \]

    \vskip 0.1in

    da cui con $x = \left( \frac{p-1}{2} \right)!$ si verifica la
    tesi.
\end{proof}

\begin{theorem}
    \label{th:primo_1_mod_4_riducibile}
    Sia $p$ un numero primo tale che $p \equiv 1 \pmod4$. Allora
    $p$ è riducibile in $\ZZi$.
\end{theorem}

\begin{proof}
    Per il \textit{Lemma \ref{lem:quadrato_mod_4}}, si ha che esiste
    un $x \in \ZZ$ tale che $p \mid x^2+1$. Se $p$ fosse irriducibile,
    dacché $\ZZi$ è un PID in quanto euclideo, $p$ sarebbe anche un
    primo di $\ZZi$. Dal momento che $x^2+1=(x+i)(x-i)$, $p$ dovrebbe
    dividere almeno uno di questi due fattori. \\

    Senza perdità di generalità, si ponga che $p \mid (x+i)$. Allora
    $\exists a+bi \in \ZZi \mid x+i=(a+bi)p$. Uguagliando le parti
    immaginarie si ottiene $bp=1$, che non ammette soluzioni, \Lightning{}. Pertanto $p$ è riducibile.
\end{proof}

\begin{corollary}[\textit{Teorema di Natale di Fermat}]
    \label{th:teorema_natale}
    Sia $p$ un numero primo tale che $p \equiv 1 \pmod4$. Allora
    $p$ è somma di due quadrati in $\ZZ$.
\end{corollary}

\begin{proof}
    Per il \textit{Teorema \ref{th:primo_1_mod_4_riducibile}},
    $p$ è riducibile in $\ZZi$. In quanto riducibile in $\ZZi$, per
    il \textit{Lemma \ref{lem:riducibile_due_quadrati}}, $p$ è allora
    somma di due quadrati.
\end{proof}

\begin{theorem}
    \label{th:primo_-1_mod_4_irriducibile}
    Sia $p$ un numero primo tale che $p \equiv -1 \pmod4$. Allora
    $p$ è irriducibile in $\ZZi$.
\end{theorem}

\begin{proof}
    Se $p$ fosse riducibile in
    $\ZZi$, per il \nameref{th:teorema_natale} esisterebbero $a$ e $b$
    in $\ZZ$ tali che $p=a^2+b^2$. Dal momento che $p$ è dispari,
    possiamo supporre, senza perdità di generalità, che
    $a$ sia pari e che $b$ sia dispari. Pertanto $a^2 \equiv 0 \pmod 4$ e $b^2 \equiv 1 \pmod 4$, dacché sono uno pari e l'altro dispari\footnote{Infatti, $0^2 \equiv 0
            \pmod4$, $1^2 \equiv 1 \pmod4$, $2^2 \equiv 4 \equiv 0 \pmod 4$,
        $3^2 \equiv 9 \equiv 1 \pmod 4$.}. Tuttavia la congruenza
    $a^2+b^2 \equiv 1 \equiv -1 \pmod4$ non è mai soddisfatta,
    \Lightning{}. Pertanto $p$ può essere solo irriducibile.
\end{proof}

\begin{note}
    Si osserva che $2=(1+i)(1-i)$. Dal momento che $\norm{1+i}^2=
        \norm{1-i}^2=2\neq1$, si deduce che nessuno dei due fattori
    è invertibile. Pertanto $2$ non è irriducibile.
\end{note}

\begin{proposition}
    \label{prop:irriducibili_zz_zzi}
    Gli unici primi $p \in \ZZ$ irriducibili in $\ZZi$ sono i primi $p$ tali
    che $p \equiv -1 \pmod4$.
\end{proposition}

\begin{proof}
    Per l'osservazione precedente, $2$ non è irriducibile in $\ZZi$,
    così come i primi congrui a $1$ in modulo $4$,
    per il \textit{Teorema \ref{th:primo_1_mod_4_riducibile}}. Al
    contrario i primi $p$ congrui a $-1$ in modulo $4$ sono
    irriducibili, per il \textit{Teorema \ref{th:primo_-1_mod_4_irriducibile}}, da cui la tesi.
\end{proof}

\begin{theorem}
    $z \in \ZZi$ è irriducibile se e solo se $z$ è un associato di un $k \in \ZZ$ tale che $k \equiv -1 \pmod 4$, o se $\norm{z}^2$ è primo.
\end{theorem}

\begin{proof} Si dimostrano le due implicazioni separatamente. \\

    ($\implies$)\; Sia $z \in \ZZi$ irriducibile. Chiaramente
    $z \mid z \overline{z} = g(z)$. Dacché $\ZZ$ è un UFD,
    $g(z)$ può decomporsi in un prodotto di primi $q_1q_2\cdots q_n$.
    Dal momento che $\ZZi$ è un PID, in quanto anello euclideo,
    $z$ deve dividere uno dei primi della fattorizzazione di
    $g(z)$. Si assuma che tale primo sia $q_i$. Allora esiste
    un $w \in \ZZi$ tale che $q_i=wz$. \\

    Se $w \in \ZZi^*$, si
    deduce che $z$ è un associato di $q_i$. Dal momento che
    $z$ è irriducibile, $q_i$, che è suo associato, è a sua
    volta irriducibile. Allora, per la \textit{Proposizione \ref{prop:irriducibili_zz_zzi}}, $q_i \equiv -1 \pmod4$.
    \\

    Altrimenti, se $w$ non è invertibile, si ha che $g(w)>g(1)$,
    ossia che $\norm{w}^2>1$. Inoltre in quanto irriducibile, anche
    $z$ non è invertibile, e quindi
    $g(z)>g(1) \implies \norm{z}^2>1$. Dalla proprietà
    moltiplicativa
    del modulo si ricava $q_i^2 = \norm{q_i}^2 = \norm{w}^2 \norm{z}^2$,
    da cui necessariamente consegue che:

    \[ \norm{w}^2=q_i, \quad \norm{z}^2=q_i, \]

    attraverso cui si verifica l'implicazione. \\

    ($\,\Longleftarrow\,\,$)\; Se $k \in \ZZ$ e $k \equiv -1 \pmod4$, per
    il \textit{Teorema \ref{th:primo_-1_mod_4_irriducibile}}, $k$ è
    irriducibile. Allora in quanto suo associato, anche $z$ è irriducibile. \\

    Altrimenti, se $\norm{z}^2$ è un primo $p$, si ponga
    $z=ab$ con $a$ e $b \in \ZZi$. Per la proprietà moltiplicativa
    del modulo, $p = \norm{z}^2 = \norm{ab}^2 = \norm{a}^2\norm{b}^2$.
    Tuttavia questo implica che uno tra $\norm{a}^2$ e $\norm{b}^2$
    sia pari a $1$, ossia che uno tra $a$ e $b$ sia invertibile,
    dacché $g(1)=1$. Pertanto $z$ è in ogni caso irriducibile.
\end{proof}

Infine si enuncia un'ultima identità inerente all'aritmetica, ma
strettamente collegata a $\ZZi$.

\subsection{L'identità di Brahmagupta-Fibonacci}

\begin{proposition}[\textit{Identità di Brahmagupta-Fibonacci}]
    \label{prop:fibonacci}
    Il prodotto di due somme di quadrati è ancora una
    somma di quadrati. In particolare:

    \[ (a^2+b^2)(c^2+d^2)=(ac-bd)^2+(ad+bc)^2. \]
\end{proposition}

\begin{proof}
    La dimostrazione altro non è che una banale verifica
    algebrica. Ciononostante è possibile risalire a questa
    identità in via alternativa mediante l'uso
    del modulo dei numeri complessi. \\

    Siano $z_1=a+bi$, $z_2=c+di \in \CC$. Allora, per le proprietà
    del modulo dei numeri complessi:

    \begin{equation}
        \label{eq:modulo_z}
        \norm{z_1}\norm{z_2}=\norm{z_1z_2}.
    \end{equation}


    Computando il prodotto tra $z_1$ e $z_2$ si ottiene:

    \[ z_1z_2 = (ac-bd) + (ad+bc)i, \]

    da cui a sua volta si ricava:

    \[ \norm{z_1z_2} = \sqrt{(ac-bd)^2 + (ad+bc)^2}, \]

    assieme a:

    \[ \norm{z_1}=\sqrt{a^2+b^2}, \quad \norm{z_2}=\sqrt{c^2+d^2}. \]

    Infine, da \eqref{eq:modulo_z}, elevando al quadrato, si deduce l'identità
    presentata:

    \begin{multline*}
        \sqrt{a^2+b^2}\sqrt{c^2+d^2}=\sqrt{(ac-bd)^2 + (ad+bc)^2} \implies (a^2+b^2)(c^2+d^2)= \\ (ac-bd)^2+(ad+bc)^2.
    \end{multline*}
\end{proof}

\begin{example}
    Si consideri $65=5 \cdot 13$. Dal momento che sia $5$
    che $13$ sono congrui a $1$ in modulo $4$, sappiamo
    già si possono scrivere entrambi come somme di due
    quadrati. Allora, dall'\nameref{prop:fibonacci},
    anche $65$ è somma di due quadrati. \\

    Infatti $5=2^2+1^2$ e $13=3^2+2^2$. Pertanto
    $65=5\cdot 13=(2\cdot3-1\cdot2)^2 + (2\cdot2+1\cdot3)^2=4^2+7^2$.
\end{example}

\section{Irriducibilità in $\ZZx$ e in $\QQx$}

\subsection{Criterio di Eisenstein e proiezione in $\ZZpx$}

Prima di studiare le irriducibilità in $\ZZ$, si guarda
alle irriducibilità nei vari campi finiti $\ZZp$, con
$p$ primo. Questo metodo presenta un vantaggio da non
sottovalutare: in $\ZZp$ per ogni grado $n$ esiste un
numero finito di polinomi monici\footnote{Si prendono in
    considerazione solo i polinomi monici dal momento che vale
    l'equivalenza degli associati: se $a$ divide $b$, allora
    tutti gli associati di $a$ dividono $b$. $\ZZp$ è infatti
    un campo, e quindi $\ZZpx$ è un anello euclideo.} -- in particolare, $p^n$ --
e quindi per un polinomio di grado $d$ è sufficiente controllare
che questo non sia prodotto di tali polinomi monici per
$1 \leq n < d$. \\

In modo preliminare, si definisce un omomorfismo fondamentale.

\begin{definition}
    Sia il seguente l'\textbf{omomorfismo di proiezione} da
    $\ZZ$ in $\ZZp$:

    \[ \hatpip : \ZZx \to \ZZpx,\, a_n x^n + \ldots + a_0 \mapsto [a_n]_p \, x^n + \ldots + [a_0]_p. \]
\end{definition}

\begin{note}
    Si dimostra facilmente che $\hatpi$ è un omomorfismo di anelli.
    Innanzitutto, $\hatpi(1) = [1]_p$. Vale chiaramente la linearità:

    \begin{multline*}
        \hatpip(a_n x^n + \ldots + a_0) + \hatpip(b_n x^n + \ldots + b_0) = [a_n]_p \, x^n + \ldots + [b_n]_p \, x^n + \ldots = \\
        = [a_n+b_n]_p \, x^n + \ldots = \hatpip(a_n x^n + \ldots + a_0 + b_n x^n + \ldots + b_0).
    \end{multline*}

    Infine vale anche la moltiplicatività:

    \begin{multline*}
        \hatpip(a_n x^n + \ldots + a_0) \hatpip(b_n x^n + \ldots + b_0) = ([a_n]_p \, x^n + \ldots)([b_n]_p \, x^n + \ldots) = \\
        = \sum_{i=0}^n \sum_{j+k=i} [a_j]_p \, [b_k]_p \, x^i
        = \sum_{i=0}^n \sum_{j+k=i} [a_j b_k]_p \, x^i
        = \hatpip\left(\sum_{i=0}^n \sum_{j+k=i} a_j b_k x^i\right) = \\
        =\hatpip\left((a_n x^n + \ldots + a_0)(b_n x^n + \ldots + b_0)\right).
    \end{multline*}
\end{note}

Prima di enunciare un teorema che si rivelerà
importante nel determinare l'irriducibilità di un
polinomio in $\ZZx$, si enuncia una definizione che
verrà ripresa anche in seguito

\begin{definition}
    Un polinomio $a_n x^n + \ldots + a_0 \in \ZZx$ si dice
    \textbf{primitivo} se $\MCD(a_n, \ldots, a_0)=1$.
\end{definition}

\begin{theorem}
    \label{th:proiezione_irriducibilità}
    Sia $p$ un primo. Sia $f(x) = a_n x^n + \ldots \in \ZZx$
    primitivo. Se $p \nmid a_n$ e
    $\hatpip(f(x))$ è irriducibile in $\ZZpx$, allora anche $f(x)$ lo
    è in $\ZZx$.
\end{theorem}

\begin{proof}
    Si dimostra la tesi contronominalmente. Sia $f(x) =
        a_nx^n + \ldots \in \ZZ[x]$ primitivo e riducibile, con
    $p \nmid a_n$. Dal momento che $f(x)$ è riducibile, esistono
    $g(x)$, $h(x)$ non invertibili tali che $f(x)=g(x)h(x)$. \\

    Si dimostra che $\deg g(x) \geq 1$. Se infatti fosse nullo,
    $g(x)$ dovrebbe o essere uguale a $\pm 1$ -- assurdo, dal
    momento che $g(x)$ non è invertibile, \Lightning{} -- o
    essere una costante non invertibile. Tuttavia, nell'ultimo
    caso, risulterebbe che $f(x)$ non è primitivo, poiché
    $g(x)$ dividerebbe ogni coefficiente del polinomio.
    Analogamente anche $\deg h(x) \geq 1$. \\

    Si consideri ora $\hatpip(f(x))=\hatpip(g(x))\hatpip(h(x))$.
    Dal momento che $p \nmid a_n$, il grado di $f(x)$ rimane costante
    sotto l'operazione di omomorfismo, ossia $\deg \hatpip(f(x)) =
        \deg f(x)$. \\

    Inoltre, poiché nessuno dei fattori di $f(x)$ è nullo, $\deg f(x) = \deg g(x) +
        \deg h(x)$. Da questa considerazione si deduce che anche i
    gradi di $g(x)$ e $h(x)$ non devono calare, altrimenti si
    avrebbe che $\deg \hatpip(f(x)) < \deg f(x)$, \Lightning{}.
    Allora $\deg \hatpip(g(x)) = \deg g(x) \geq 1$,
    $\deg \hatpip(h(x)) = \deg h(x) \geq 1$. \\

    Poiché $\deg \hatpip(g(x))$ e $\deg \hatpip(h(x))$ sono
    dunque entrambi non nulli, $\hatpip(g(x))$ e $\hatpip(h(x))$
    non sono invertibili\footnote{Si ricorda che $\ZZpx$
        è un anello euclideo. Pertanto, non avere lo stesso grado
        dell'unità equivale a non essere invertibili.}. Quindi
    $f(x)$ è prodotto di non invertibili, ed è dunque riducibile.

\end{proof}

\begin{theorem}[\textit{Criterio di Eisenstein}]
    \label{th:eisenstein}
    Sia $p$ un primo.
    Sia $f(x) = a_n x^n + \ldots + a_0 \in \ZZx$ primitivo tale che:

    \begin{enumerate}[ (1)]
        \item $p \nmid a_n$,
        \item $p \mid a_i$, $\forall i \neq n$,
        \item $p^2 \nmid a_0$.
    \end{enumerate}

    Allora $f(x)$ è irriducibile in $\ZZx$.
\end{theorem}

\begin{proof}
    Si ponga $f(x)$ riducibile e sia pertanto $f(x)=g(x)h(x)$ con
    $g(x)$ e $h(x)$ non invertibili. Analogamente a come visto
    per il \textit{Teorema \ref{th:proiezione_irriducibilità}}, si
    desume che $\deg g(x)$, $\deg h(x) \geq 1$. \\

    Si applica l'omomorfismo di proiezione in $\ZZpx$:

    \[ \hatpip(f(x))=\underbrace{[a_n]_p}_{\neq 0} x_n, \]

    da cui si deduce che $\deg \hatpip(f(x)) = \deg f(x)$. \\

    Dal momento che $\hatpip(f(x))=\hatpip(g(x))\hatpip(h(x))$ e
    che $\ZZpx$, in quanto campo, è un dominio,
    necessariamente sia $\hatpip(g(x))$ che $\hatpip(h(x))$
    sono dei monomi. \\

    Inoltre, sempre in modo analogo a come visto per il \textit{Teorema
        \ref{th:proiezione_irriducibilità}}, sia $\deg \hatpip(g(x))$
    che $\deg \hatpip(h(x))$ sono maggiori o uguali ad $1$. \\

    Combinando questo risultato col fatto che questi due fattori
    sono monomi, si desume che
    $\hatpip(g(x))$ e $\hatpip(h(x))$ sono monomi di grado positivo.
    Quindi $p$ deve dividere entrambi i termini noti di $g(x)$ e
    $h(x)$, e in particolare $p^2$ deve dividere il loro prodotto,
    ossia $a_0$. Tuttavia questo è un assurdo, \Lightning{}.
\end{proof}

\begin{note}
    Si consideri $x^k-2$, per $k \geq 1$.
    Per il \nameref{th:eisenstein},
    considerando come primo $p=2$, si verifica che
    $x^k-2$ è sempre irriducibile. Pertanto, per ogni
    grado di un polinomio esiste almeno un irriducibile --
    a differenza di come invece avviene in $\RRx$ o in $\CCx$.
\end{note}

\begin{theorem}
    Sia $f(x) \in \ZZx$ primitivo e sia $a \in \ZZ$. Allora $f(x)$ è
    irriducibile se e solo se $f(x+a)$ è irriducibile.
\end{theorem}

\begin{proof}
    Si dimostra una sola implicazione, dal momento che l'implicazione
    contraria consegue dalle stesse considerazioni poste
    studiando prima $f(x+a)$ e poi $f(x)$. \\

    Sia $f(x)=a(x)b(x)$ riducibile, con $a(x)$, $b(x) \in \ZZx$ non
    invertibili. Come già visto per il \textit{Teorema
        \ref{th:proiezione_irriducibilità}}, $\deg a(x)$, $\deg b(x) \geq 1$. \\

    Allora chiaramente $f(x+a)=g(x+a)h(x+a)$, con $\deg g(x+a) =
        \deg g(x) \geq 1$, $\deg h(x+a) = \deg h(x) \geq 1$. Pertanto
    $f(x+a)$ continua a essere riducibile, da cui la tesi.
\end{proof}

\begin{example}
    Si consideri $f(x) = x^{p-1}+\ldots+x^2+x+1 \in \ZZx$, dove
    tutti i coefficienti del polinomio sono $1$. Si verifica che:

    \[ f(x+1)=\frac{(x+1)^p-1}x = p+\binom{p}{2}x+\ldots+x^{p-1}. \]

    Allora, per il \nameref{th:eisenstein} con $p$, $f(x+1)$ è
    irriducibile. Pertanto anche $f(x)$ lo è.
\end{example}

\subsection{Alcuni irriducibili di $\ZZ_2[x]$}

Tra tutti gli anelli $\ZZpx$, $\ZZ_2[x]$ ricopre sicuramente
un ruolo fondamentale, dal momento che è il meno costoso
computazionalmente da analizzare, dacché $\ZZ_2$ consta
di soli due elementi. Pertanto si computano adesso gli
irriducibili di $\ZZ_2[x]$ fino al quarto grado incluso, a meno
di associati. \\

Sicuramente $x$ e $x+1$ sono irriducibili, dal momento che sono di
primo grado. I polinomi di secondo grado devono dunque essere
prodotto di questi polinomi, e pertanto devono avere o $0$ o
$1$ come radice: si verifica quindi che $x^2+x+1$ è l'unico
polinomio di secondo grado irriducibile. \\

Per il terzo grado vale ancora lo stesso principio, per cui
$x^3+x^2+1$ e $x^3+x+1$ sono gli unici irriducibili di tale grado.
Infine, per il quarto grado, i polinomi riducibili soddisfano
una qualsiasi delle seguenti proprietà:

\begin{itemize}
    \item $0$ e $1$ sono radici del polinomio,
    \item il polinomio è prodotto di due polinomi irriducibili di
          secondo grado.
\end{itemize}

Si escludono pertanto dagli irriducibili i polinomi non omogenei --
che hanno sicuramente $0$ come radice --, e i polinomi con $1$ come
radice, ossia $x^4+x^3+x+1$,\ \
$x^4+x^3+x^2+1$, e $x^4+x^2+x+1$. Si esclude anche
$(x^2+x+1)^2 = x^4+x^2+1$. Pertanto gli unici irriducibili di
grado quattro sono $x^4+x^3+x^2+x+1$,\ \ $x^4+x^3+1$,\ \  $x^4+x+1$. \\

Tutti questi irriducibili sono raccolti nella seguente tabella:

\begin{itemize}
    \item (grado 1) $x$, $x+1$,
    \item (grado 2) $x^2+x+1$,
    \item (grado 3) $x^3+x^2+1$, $x^3+x+1$,
    \item (grado 4) $x^4+x^3+x^2+x+1$,\ \ $x^4+x^3+1$,\ \ $x^4+x+1$.
\end{itemize}

\begin{example}
    Il polinomio $51x^3+11x^2+1 \in \ZZx$ è primitivo dal momento
    che $\MCD(51, 11, 1)=1$. Inoltre, poiché $\hatpi_2(51x^3+11x^2+1)=
        x^3+x+1$ è irriducibile, si deduce che anche $51x^3+11x^2+1$ lo
    è per il \textit{Teorema \ref{th:proiezione_irriducibilità}}.
\end{example}

\subsection{Teorema delle radici razionali e lemma di Gauss}

Si enunciano in questa sezione i teoremi più importanti per
lo studio dell'irriducibilità dei polinomi in $\QQx$ e
in $\ZZx$, a partire dai due teoremi più importanti: il
classico \nameref{th:radici_razionali} e il \nameref{th:lemma_gauss},
che si pone da ponte tra l'analisi dell'irriducibilità in $\ZZx$ e
quella in $\QQx$.

\begin{theorem}[\textit{Teorema delle radici razionali}]
    \label{th:radici_razionali}
    Sia $f(x) = a_n x^n + \ldots + a_0 \in \ZZx$. Abbia $f(x)$
    una radice razionale. Allora, detta tale radice $\frac{p}{q}$,  già ridotta ai minimi termini, questa è tale che:

    \begin{enumerate}[ (i.)]
        \item $p \mid a_0$,
        \item $q \mid a_n$.
    \end{enumerate}
\end{theorem}

\begin{proof}
    Poiché $\frac{p}{q}$ è radice, $f\left(\frac{p}{q}\right)=0$, e
    quindi si ricava che:

    \[ a_n \left( \frac{p}{q} \right)^n + \ldots + a_0 = 0 \implies
        a_n p^n = -q( \ldots + a_0 q^{n-1}). \]

    \vskip 0.1in

    Quindi $q \mid a_n p^n$. Dal momento che $\MCD(p, q)=1$, si
    deduce che $q \mid a_n$. \\

    Analogamente si ricava che:

    \[ a_0 q^n = -p(a_n p^{n-1} + \ldots). \]

    \vskip 0.1in

    Pertanto, per lo stesso motivo espresso in precedenza,
    $p \mid a_0$, da cui la tesi.
\end{proof}

\begin{theorem}[\textit{Lemma di Gauss}]
    \label{th:lemma_gauss}
    Il prodotto di due polinomi primitivi in $\ZZx$ è anch'esso primitivo.
\end{theorem}

\begin{proof}
    Siano $g(x) = a_m x^m + \ldots + a_0$ e $h(x) = b^n x^n + \ldots + b_0$ due polinomi primitivi in $\ZZx$. Si assuma che $f(x)=g(x)h(x)$
    non sia primitivo. Allora esiste un $p$ primo che divide tutti i
    coefficienti di $f(x)$. \\

    Siano $a_s$ e $b_t$ i più piccoli coefficienti non divisibili
    da $p$ dei rispettivi polinomi. Questi sicuramente esistono,
    altrimenti $p$ dividerebbe tutti i coefficienti, e quindi
    o $g(x)$ o $h(x)$ non sarebbe primitivo, \Lightning{}. \\

    Si consideri il coefficiente di $x^{s+t}$ di $f(x)$:

    \[c_{s+t} = \sum_{j+k=s+t} a_j b_k = \underbrace{a_0 b_{s+t} + a_1 b_{s+t-1} + \ldots}_{\equiv \, 0 \pmod p} + a_s b_t + \underbrace{a_{s+1}b_{t-1} + \ldots}_{\equiv \, 0 \pmod p},\]

    dal momento che $p \mid c_{s+t}$, si deduce che $p$ deve dividere
    anche $a_sb_t$, ossia uno tra $a_s$ e $b_t$, che è assurdo, \Lightning{}. Quindi $f(x)$ è primitivo.

\end{proof}

\begin{theorem}[\textit{Secondo lemma di Gauss}]
    \label{th:lemma_gauss_2}
    Sia $f(x) \in \ZZx$. Allora $f(x)$ è irriducibile in $\ZZx$
    se e solo se $f(x)$ è irriducibile in $\QQx$ ed è primitivo.
\end{theorem}

\begin{proof} Si dimostrano le due implicazioni separatamente. \\

    ($\implies$)\; Si dimostra l'implicazione contronominalmente,
    ossia mostrando che se $f(x)$ non è primitivo o se è
    riducibile in $\QQx$, allora $f(x)$ è riducibile in $\ZZx$. \\

    Se $f(x)$ non è primitivo, allora
    $f(x)$ è riducibile in $\ZZx$. Sia quindi $f(x)$ primitivo
    e riducibile in $\QQx$, con $f(x)=g(x)h(x)$,
    $g(x)$, $h(x) \in \QQx \setminus \QQx^*$. \\

    Si descrivano $g(x)$ e $h(x)$ nel seguente modo:

    \[ g(x)=\frac{p_m}{q_m} x^m + \ldots + \frac{p_0}{q_0}, \quad \MCD(p_i, q_i)=1 \; \forall 0 \leq i \leq m, \]

    \[ h(x)=\frac{s_n}{t_n} x^n + \ldots + \frac{s_0}{t_0}, \quad
        \MCD(s_i, t_i)=1 \; \forall 0 \leq i \leq n. \]

    \vskip 0.1in

    Si definiscano inoltre le seguenti costanti:

    \[ \alpha = \frac{\mcm(q_m, \ldots, q_0)}{\MCD(p_m, \ldots, p_0)}, \quad \beta = \frac{\mcm(t_n, \ldots, t_0)}{\MCD(s_n, \ldots, s_0)}. \]

    \vskip 0.1in

    Si verifica che sia $\hat{g}(x)=\alpha g(x)$ che
    $\hat{h}(x)=\beta h(x)$ appartengono a $\ZZx$ e che entrambi
    sono primitivi. Pertanto $\hat{g}(x) \hat{h}(x) \in \ZZx$. \\

    Si descriva $f(x)$ nel seguente modo:

    \[ f(x)=a_k x^k + \ldots + a_0, \quad \MCD(a_k,\ldots,a_0)=1. \]

    \vskip 0.1in

    Sia $\alpha \beta = \frac{p}{q}$ con $\MCD(p,q)=1$, allora:

    \[\hat{g}(x) \hat{h}(x) = \alpha \beta f(x) = \frac{p}{q} (a_k x^k + \ldots + a_0), \]

    da cui, per far sì che $\hat{g}(x) \hat{h}(x)$ appartenga
    a $\ZZx$, $q$ deve necessariamente dividere tutti i
    coefficienti di $f(x)$. Tuttavia $f(x)$ è primitivo, e quindi
    $q=\pm 1$. Pertanto $\alpha \beta = \pm p \in \ZZ$. \\

    Infine, per il \nameref{th:lemma_gauss}, $\alpha \beta f(x)$
    è primitivo, da cui $\alpha \beta = \pm 1$. Quindi
    $f(x) = \pm \hat{g}(x) \hat{h}(x)$ è riducibile. \\

    ($\,\Longleftarrow\,\,$)\; Se $f(x)$ è irriducibile in $\QQx$
    ed è primitivo, sicuramente $f(x)$ è irriducibile anche in
    $\ZZx$. Infatti, se esiste una fattorizzazione in
    irriducibili in $\ZZx$, essa non include alcuna costante
    moltiplicativa dal momento che $f(x)$ è primitivo, e quindi
    esisterebbe una fattorizzazione in irriducibili anche in $\QQx$.
\end{proof}

\end{document}
